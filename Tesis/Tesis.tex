\documentclass[twoside, 12pt]{article}

\usepackage[sc]{mathpazo} % Use the Palatino font
\usepackage[T1]{fontenc} % Use 8-bit encoding that has 256 glyphs
\linespread{1.5} % Line spacing - Palatino needs more space between lines

%\usepackage[twoside,width=16cm,height=24cm,left=3cm]{geometry}
\usepackage[hmarginratio=1:1,top=20mm,width=20cm,height=23.7cm,columnsep=15pt]{geometry} % Document margins
\usepackage{multicol} % Used for the two-column layout of the document
\usepackage[hang, small,labelfont=bf,up,textfont=it,up]{caption} % Custom captions under/above floats in tables or figures
\usepackage{booktabs} % Horizontal rules in tables
\usepackage{float} % Required for tables and figures in the multi-column environment - they need to be placed in specific locations with the [H] (e.g. \begin{table}[H])
\usepackage{hyperref} % For hyperlinks in the PDF

%----------- Agregados para el caso de ustedes -------------------------------
\usepackage[spanish]{babel}% idioma castellano
\usepackage[utf8]{inputenc}% esto es para poder poner los tildes directamente. Puede que cambie de versión a versión de sistema operativos (más información en http://www.aq.upm.es/Departamentos/Fisica/agmartin/webpublico/latex/FAQ-CervanTeX/FAQ-CervanTeX-6.html )
\usepackage{graphicx} % para insertar figuras
\usepackage{subfigure} % para insertar figuras dentro de figuras
\usepackage{times} % plataforma
\usepackage{amsmath} % --para ecuaciones y algunos símbolos
\usepackage{wrapfig,lipsum}
\usepackage{listings}
\usepackage{color}
\usepackage{braket}
\usepackage{amsthm}
\usepackage{cancel}

\definecolor{dkgreen}{rgb}{0,0.6,0}
\definecolor{gray}{rgb}{0.5,0.5,0.5}
\definecolor{mauve}{rgb}{0.58,0,0.82}

\lstset{frame=tb,
	language=C++,
	aboveskip=3mm,
	belowskip=3mm,
	showstringspaces=false,
	columns=flexible,
	basicstyle={\small\ttfamily},
	numbers=none,
	numberstyle=\tiny\color{gray},
	keywordstyle=\color{blue},
	commentstyle=\color{dkgreen},
	stringstyle=\color{mauve},
	breaklines=true,
	breakatwhitespace=true,
	tabsize=3
}
% ---------------------- -----------------------------------------------------

\usepackage{lettrine} % The lettrine is the first enlarged letter at the beginning of the text
\usepackage{paralist} % Used for the compactitem environment which makes bullet points with less space between them
\usepackage[T1]{fontenc}					%para poder usar tildes sin problemas

\usepackage{mathrsfs}
% Abreviaturas
%\newcommand\RR{\mathbb{R}}

\newtheorem{theorem}{Theorem}
\providecommand{\dpart}[2]{\frac{\partial#1}{\partial#2}}
\providecommand{\poisson}[2]{\left\{#1; #2 \right\}_{Poisson}}

\graphicspath{{Imagenes/}}

\begin{document}
 
\section{Introducción}
 \subsection{Gas de Fermi}{\label{sec:intro_fermi_gas}}

En los sistemas cuánticos, la indistinguibilidad de las partículas juega un rol esencial.
Esta indistinguibilidad obliga a todos los autoestados $\ket{\Psi}$ del sistema a ser autoestados del operador permutación de 2 partículas $\hat{P}_{ij}$ (permuta la $i$-esima con la $j$-esima), cuyos únicos autovalores son $\pm 1$.
De esta manera, se asegura la invarianza de los observables ante $\hat{P}_{ij}$
\[\bra{\Psi'}\hat{A}\ket{\Psi'} = \bra{\Psi}\hat{P}^\dagger\hat{A}\hat{P}\ket{\Psi} = (\bra{\Psi}\pm)\hat{A}(\pm\ket{\Psi}) = \bra{\Psi}\hat{A}\ket{\Psi}\]

Se llaman \textit{bosones} a las partículas cuyas funciones de onda tienen autovalor positivo ante cualquier permutación de 2 partículas (simétricas $\hat{P}_{ij}\ket{\Psi} = \ket{\Psi}$) y \textit{fermiones} a aquellas con autovalor negativo (antisimétricas $\hat{P}_{ij}\ket{\Psi} = -\ket{\Psi}$).
De esta propiedad de los fermiones surge el principio de exclusión de Pauli, que prohíbe la ocupación de un mismo estado por más de un fermion.
De otra forma, sería imposible lograr que la función de onda sea antisimétrica, dado que si tuviésemos dos partículas $i$ y $j$ en el mismo estado, la permutación $\hat{P}_{ij}$ mantendría
$\ket{\Psi}$ invariante en lugar de cambiar su signo.

Según el teorema de spin-estadística de Dirac, los fermiones poseen spin semi-entero $s=l+1/2$, razón por la que los nucleones (protones y neutrones) y los electrones son fermiones.
La exclusión de Pauli es fundamental a la hora de comprender la dinámica de fermiones a bajas temperaturas.
Dada esta exclusión, aún a temperatura cero resulta imposible que todos los fermiones condensen a sus estados (de única partícula) de menor energía. 
Esto hace que los distintos niveles de energía se ``llenen'' uno a uno, lo cual inevitablemente genera sistemas con mayor energía total.
Gracias a esto puede explicarse desde la conducción eléctrica en metales\cite[pp. 247]{BOOK:PATHRIA} hasta la \textit{presión de degeneración} 
(presión no nula a temperatura cero) que evita el colapso de las enanas blancas \cite[pp2. 59]{BOOK:PATHRIA}.

El sistema de fermiones no interactuantes es el más simple, cuya estadística es bien conocida, siendo su función de partición gran canónica y ocupación media\cite[pp. 231]{BOOK:PATHRIA}
\begin{equation}
 \frac{PV}{k_BT} \equiv \log Z = \sum_\varepsilon \log(1+ze^{-\beta\varepsilon})
\end{equation}
\begin{equation}
 N = \sum_\varepsilon \langle n_\varepsilon\rangle = \sum_\varepsilon \frac{1}{z^{-1}e^{\beta\varepsilon}+1}
\end{equation}
con $z=e^{\beta\mu}$ la fugacidad, $\beta=1/k_BT$ y $\mu$ el potencial químico del sistema.
Los niveles de energía de una partícula $\varepsilon=p^2/2m$ están discretizados y dependen del volumen $V$, con $\Delta\varepsilon\sim V^{-1/3}$.

En el límite termodinámico ($V\to\infty$), la diferencia entre energías $\Delta\varepsilon\to 0$ y podemos transformar la sumatoria en una integral en el espacio de fases.
Realizando las integrales triviales, resulta
\begin{equation}{\label{eq:Z_cont}}
 \frac{PV}{k_BT} \equiv \log Z = \frac{4\pi V}{h^3} \int_0^\infty \log(1+ze^{-\beta p^2/2m})p^2dp = \frac{gV}{\lambda^3}f_{5/2}(z)
\end{equation}
\begin{equation}{\label{eq:N_cont}}
 N = \frac{4\pi V}{h^3} \int_0^\infty \frac{1}{z^{-1}e^{\beta p^2/2m}+1} p^2dp = \frac{gV}{\lambda^3}f_{3/2}(z)
\end{equation}
donde $g$ es un factor de degeneración asociado a grados de libertad internos (spin, por ejemplo) y  $\lambda$ es la longitud de onda térmica
\[ \lambda = \sqrt{ \frac{2\pi\hbar^2}{mk_BT} } \]
y las $f_\nu(z)$ son las funciones de Fermi-Dirac definidas según
\[ f_\nu(z) = \frac{1}{\Gamma(\nu)}\int_0^\infty \frac{x^{\nu-1}}{z^{-1}e^x+1} = \sum_{l=1}^\infty (-1)^{l+1}\frac{z^l}{l^\nu}\]

Para un sistema de número de partículas, volumen y temperatura constante (o NVT), debemos obtener la fugacidad $z$ invirtiendo \eqref{eq:N_cont}, lo que introduce el problema de computar las funciones de Fermi-Dirac $f_\nu(z)$.
Al ser series alternadas, la convergencia de las sumas parciales es muy lenta y debemos recurrir a otros métodos, que detallamos en el Apéndice \ref{ap:func_fermi}.
Independientemente de esto, puede verse que obtendremos un $\mu(N/V,T) = \mu(\rho, T)$, cuya dependencia en $\rho$ en lugar de $N$ y $V$ por separado es razonable al ser $\mu$ una magnitud intensiva.

En particular, podemos ver en \eqref{eq:N_cont} la aparición del producto $\lambda^3\rho \sim T^{-3/2}\rho$ a la hora de despejar $f_{3/2}(z)$, que resulta central para las propiedades del sistema.
Para $\lambda^3\rho \sim T^{-3/2}\rho\to0$ tenemos $f_{3/2}(z)$ pequeño, por lo que a primer orden tenemos $f_{3/2}(z)\approx z\to 0$.
Reemplazando en \eqref{eq:Z_cont} y \eqref{eq:N_cont} y dividiendo ambas ecuaciones, tenemos 
\[ \frac{PV}{Nk_BT} = \frac{f_{5/2}(z)}{f_{3/2}(z)} \approx 1 \]
que resulta la ecuación de estado de un gas ideal clásico, lo cual esperamos que ocurra a alta temperatura (o baja densidad).
Por lo tanto, la noción de alta temperatura (o baja densidad) viene dada por el valor del producto $\lambda^3\rho$, tal que $T\to\infty$ resulta equivalente a $\rho\to0$.
El caso opuesto $\lambda^3\rho\to\infty$ corresponde a baja temperatura (o alta densidad), donde las propiedades cuánticas son más apreciables.

El integrando de \eqref{eq:N_cont} puede considerarse como la densidad de partículas $f(p)$ con impulso de módulo $p$.
Esta distribución de impulsos es característica de un gas de fermiones isótropo y homogéneo.
Equivalentemente, podríamos analizar la distribución de energías cinéticas $\varepsilon=p^2/2m$ mediante un cambio de variables sobre la integral de \eqref{eq:N_cont}
($d\varepsilon = pdp/m = \sqrt{2m\varepsilon}dp/m$)

\[ N = \frac{4\pi V}{h^3} \int_0^\infty \frac{1}{e^{\beta(\varepsilon-\mu)}+1} 2m\varepsilon\frac{m}{\sqrt{2m\varepsilon}}d\varepsilon =
\int_0^\infty \frac{4\pi V\sqrt{2m^3\varepsilon}}{h^3}\frac{1}{e^{\beta(\varepsilon-\mu)}+1} d\varepsilon\]

Por lo tanto, la distribución de impulsos toma la forma
\begin{equation}{\label{eq:dist_FD}}
 f_{FD}(\varepsilon) = \frac{4\pi V \sqrt{2m^3\varepsilon}}{h^3}\frac{1}{e^{\beta(\varepsilon-\mu)}+1} = \frac{G(\varepsilon)}{e^{\beta(\varepsilon-\mu)}+1}
 %= N\frac{V}{\lambda^3}\sqrt{\frac{4\beta^3\varepsilon}{\pi}}\frac{1}{e^{\beta(\varepsilon-\mu)}+1}
\end{equation}
donde $G(\varepsilon)$ es la degeneración de estados con energía $\varepsilon$ y asumimos $\mu=\mu(\rho,T)$.

En el límite de altas temperaturas (o baja densidad), esperamos que el sistema pierda sus propiedades cuánticas y recuperemos el gas clásico, donde las partículas
se vuelven distinguibles nuevamente.
Además de recuperar la ecuación de estado del gas ideal, esperamos que en el límite $\lambda^3\rho\to0$, la distribución de energías cinéticas tienda a la de un gas clásico.
Para un sistema de partículas clásico interactuando con potenciales dependientes de la posición $U(\mathbf{q}_1,..,\mathbf{q}_N)$, la distribución de energías cinéticas resulta
corresponde a la de Boltzmann (ver \textbf{Apéndice \ref{ap:boltzmann}}).
\begin{equation}{\label{eq:dist_MB}}
 f_{MB}(\varepsilon) = N\sqrt{\frac{4\beta^3\varepsilon}{\pi}}e^{-\beta\varepsilon}
\end{equation}



\subsection{Potencial de Pauli}{\label{sec:intro_pauli}}

Dado que la distribución \eqref{eq:dist_MB} es común a todo sistema de partículas distinguibles (o clásicas) cuyos Hamiltonianos cuya dependencia con el momento
sea a través de la energía cinética (ver  \textbf{Apéndice \ref{ap:boltzmann}})
\[ H(\mathbf{q}_1,..,\mathbf{q}_N,\mathbf{p}_1,..,\mathbf{p}_N) = \sum_i \frac{p_i^2}{2m} + U(\mathbf{q}_1,..,\mathbf{q}_N)\]
la única forma de obtener una distribución de energías distinta con partículas distinguibles es introduciendo potenciales dependientes de momentos al Hamiltoniano.

Esto puede hacerse de varias maneras, pero en particular buscamos una cuya distribución resulte similar a \eqref{eq:dist_FD}.
Para esto, necesitamos poder expresar la noción de Exclusión de Pauli con un potencial de interacción de dos partículas $V(\mathbf{q}_1,..,\mathbf{q}_N;\mathbf{p}_1,..,\mathbf{p}_N)$.
Esto lo haremos imponiendo un ``costo energético'' cada vez que 2 partículas tengan un $\mathbf{q}, \mathbf{p}$ similar.
Buscamos lograr esto agregando al Hamiltoniano un potencial de interacción de 2 partículas que llamaremos \textit{potencial de Pauli} definido según

\begin{equation}{\label{eq:def_int_pauli}}
 V_P(\mathbf{q}_1,\mathbf{q}_2;\mathbf{p}_1,\mathbf{p}_2) = De^{-\frac{1}{2}\left( \frac{|\mathbf{q}_1-\mathbf{q}_2|^2}{q_o^2} +\frac{|\mathbf{p}_1-\mathbf{p}_2|^2}{p_o^2} \right)}
\end{equation}
donde surge la noción de distancia reducida en el espacio de fases $s^2 = \frac{|\mathbf{q}_1-\mathbf{q}_2|^2}{q_o^2} +\frac{|\mathbf{p}_1-\mathbf{p}_2|^2}{p_o^2}$.

Podemos visualizar los parámetros $q_o$ y $p_o$ como los ejes de un elipsoide en $\mathbb{R}^6$ (el espacio de fases) donde $s^2 \leq 1$ y el costo energético de cualquier partícula
que ingrese es $\approx D$.
A energías (o temperaturas) bajas, este costo energético puede ser suficientemente alto como para evitar la superposición de estos elipsoides en el espacio de fases, generando efectivamente
la noción de exclusión de Pauli: dos partículas no pueden estar en el mismo estado $(\mathbf{q}, \mathbf{p})$.
Es importante notar que este potencial no es de núcleo duro y, por lo tanto, es posible que dos partículas se superpongan para energías suficientemente altas donde $D$ sea despreciable.
Esto es deseable dado que para energías (o temperaturas) altas esperamos recuperar el gas clásico de Maxwell-Boltzmann al desaparecer $V_P$.

Este potencial fue inicialmente propuesto por Wiletz \cite{Wilets1977} y continuado por Dorso \cite{Dorso1987, Dorso1988, Dorso1993} y utilizado en múltiples trabajos desde entonces\cite{Maruyama2012, Taruna2008}.
Los parámetros $D$, $q_o$ y $p_o$ varían en cada trabajo, adaptándose según la característica que se busque analizar.
Existen formulaciones alternativas como la de Piekarewicz \cite{Taruna2008} que proponen otra forma de abordar la exclusión de Pauli; exclusión en $q$ y en $p$ independientemente.

La dependencia del potencial con el impulso trae aparejada una serie de dificultades a la hora de hacer simulaciones de Dinámica Molecular.
Más allá de la inmediata dificultad para evolucionar el sistema (que analizaremos en la sección que sigue), surge la dificultad de definir algunas magnitudes termodinámicas del
sistema como la presión $P$ y la temperatura $T$.
Para un potencial dependiente de momentos, la temperatura $T$ no resulta la energía cinética media del sistema.
El teorema del virial (ver Apéndice \ref{ap:teo_virial}) nos permite calcular ambas magnitudes según
\begin{equation}{\label{eq:virial_T}}
k_B T =  \frac{1}{3N}\left<\sum_{i=1}^{3N} \dot{q}_ip_i \right>
=  \frac{1}{3N}\left<\sum_{i=1}^{3N} \frac{p_i^2}{m} \right> - \frac{1}{3V}\left< \sum_{i=1}^{3N}\sum_{j\neq i} \dpart{V_{ij}^{(P)}}{p_i}p_i \right>
\end{equation}

\begin{equation}{\label{eq:virial_P}}
P = \frac{1}{3V}\left< \sum_{i=1}^{3N} \dot{q}_ip_i + \dot{p}_iq_i \right>
= \frac{1}{3V}\left< \sum_{i=1}^{3N} \frac{p_i^2}{m} \right> - \frac{1}{3V}\left< \sum_{i=1}^{3N}\sum_{j\neq i} \dpart{V_{ij}^{(P)}}{q_i}q_i+\dpart{V_{ij}^{(P)}}{p_i}p_i \right>
\end{equation}

\subsection{Integración simpléctica de sistemas hamiltonianos}

La evolución de las $d$ coordenadas generalizadas $q \equiv (q_1,..,q_d)\in\mathbb{R}^{d}$ y $p \equiv (p_1,..,p_d)\in\mathbb{R}^{d}$ está definida por el Hamiltoniano
$H(q_1,..,q_d,p_1,..,p_d) \equiv H(q,p)$  a través de las ecuaciones de Hamilton
\begin{align*}
 \dot{q} &= \dpart{H}{p}(q,p) \\
 \dot{p} &= -\dpart{H}{q}(q,p)
\end{align*}
que podemos compactar definiendo $y=(p,q)\in\mathbb{R}^{2d}$ y la matriz $J = \begin{pmatrix}0 & \mathbb{I} \\-\mathbb{I} & 0\end{pmatrix}$ (con la propiedad $J^{-1} = J^T = -J$)
\begin{equation}{\label{eq:ec_hamilton}}
 \dot{y} = J^{-1}\nabla H(y)
\end{equation}

Para un Hamiltoniano que no depende explícitamente del tiempo, la ecuación de evolución \eqref{eq:ec_hamilton} asegura la conservación de $H(q,p)$ en el tiempo
\[ \frac{dH}{dt}(y) = \nabla H(y)\cdot \dot{y} = \nabla H(y)^T J^{-1}\nabla H(y) = 0\]
donde usamos que $(a,b)J^{-1}\begin{pmatrix}a\\b\end{pmatrix} = (b, -a)\begin{pmatrix}a\\b\end{pmatrix} = 0$ $\forall a,b$.
Esto no es más que la conocida conservación de la energía.

\subsubsection{Transformaciones simplécticas}{\label{sec:trans_simp}}

Además de conservar la energía, los sistemas hamiltonianos tienen propiedades asociadas a la conservación del área en el espacio de fases\cite[pp. 183-185]{BOOK:SPR_INT} de $y$.
Supongamos dos vectores
\[ \mu = \begin{pmatrix} \mu^p\\ \mu^q \end{pmatrix} \qquad \eta = \begin{pmatrix} \eta^p\\ \eta^q \end{pmatrix}  \]
con $\mu^p, \mu^q, \eta^p, \eta^q\in\mathbf{R}^{d}$ las ``componentes $p$ y $q$'' de estos vectores.
Definimos el mapeo bilinear $\omega(\mu,\eta)$ como la suma de las áreas orientadas de cada par de componentes $p_i,q_i$
\begin{equation}
 \omega(\mu, \eta) = \sum_{i=1}^{3N} \det \begin{pmatrix} \mu^p_i & \eta^p_i \\ \mu^q_i & \eta^q_i \end{pmatrix} = \sum_{i=1}^{3N} \mu^p_i\eta^q_i - \mu^q_i\eta^p_i = \mu^T J \eta
\end{equation}

Con esta noción de área, decimos que un mapeo diferenciable $g:U\subseteq\mathbb{R}^{2d}\to\mathbb{R}^{2d}$ es \textit{simpléctico} si su matriz jacobiana $g'(p,q)$ cumple alguna de las condiciones
equivalentes
\begin{equation}{\label{eq:def_trans_simp}}
g'(p,q)^T J g'(p,q) = J \qquad \text{ o } \qquad \omega(g'(p,q)\mu, g'(p,q)\eta) = \omega(\mu, \eta)
\end{equation}

Para visualizar esto, supongamos $M$ una subvariedad de dimensión 2 de $U$ parametrizada por alguna función suave $\psi(s,t)$ tal que $M=\psi(K)$ para $K\subseteq\mathbb{R}^2$.
Podemos considerar a $M$ como la unión de infinitos paralelogramos infinitesimales definidos por los vectores
\[ \dpart{\psi}{s}(s,t)ds \quad \text{ y } \quad \dpart{\psi}{t}(s,t)dt \]

Sumando sobre todos estos paralelogramos, obtenemos el área de $M$ como
\begin{equation}{\label{eq:area_orien_simp}}
 \Omega(M) = \iint_K \omega\left(\dpart{\psi}{t}(s,t), \dpart{\psi}{s}(s,t)\right) ds dt
\end{equation}
y resulta inmediato ver que este volumen es invariante ante una transformación simpléctica $g$.

\begin{theorem}{\label{teo:preservacion_vol}}
 Sea un mapeo diferenciable $g:U\subseteq\mathbb{R}^{2d}\to\mathbb{R}^{2d}$ simpléctico. Luego, preserva $\Omega(M)$
 \[ \Omega(g(M)) = \Omega(M) \]
 para toda variedad $M\subseteq\mathbb{R}^2$ que pueda representarse como la imagen de una función diferenciable $\psi$.
\end{theorem}
\begin{proof}
\begin{align*}
 \Omega(g(M)) &=  \iint_K \omega\left(\dpart{(g\circ\psi)}{t}(s,t), \dpart{(g\circ\psi)}{s}(s,t)\right) ds dt \\
&= \iint_K \omega\left(g'(\psi(s,t))\dpart{\psi}{t}(s,t), g'(\psi(s,t))\dpart{\psi}{s}(s,t)\right) ds dt \\
&= \iint_K \omega\left(\dpart{\psi}{t}(s,t), \dpart{\psi}{s}(s,t)\right) ds dt = \Omega(M) 
\end{align*}
donde utilizamos la propiedad \eqref{eq:def_trans_simp} de una $g$ simpléctica.
\end{proof}

Finalmente, nos basta ver que el mapeo de evolución temporal $\varphi_t(p_o,q_o) = (p(t;p_o,q_o), q(t;p_o,q_o))$ también es simpléctico; donde $p(t;p_o,q_o), q(t;p_o,q_o)$ son las soluciones de \eqref{eq:ec_hamilton}
con condiciones iniciales $p(0)=p_o$ y $q(0)=q_o$.
Reemplazando $y$ por $\varphi_t(p_o,q_o)$ en \eqref{eq:ec_hamilton} y derivando respecto de $y_o=(p_o,q_o)$ tenemos
\begin{align*}
 \dpart{\dot{\varphi_t}}{y_o} &= J^{-1}\dpart{}{y_o} \nabla H(\varphi_t) \\
 \frac{d}{dt}\left(\dpart{\varphi_t}{y_o}\right) &= J^{-1} \nabla^2 H(\varphi_t)\dpart{\varphi_t}{y_o}
\end{align*}

donde $\nabla^2 H$ es la matriz hessiana de $H$ y, cabe destacar, es simétrica. Esta propiedad es esencial, pues de ella se deriva el siguiente teorema

\begin{theorem}{\label{teo:preservacion_evol}}
  Sea $H(p,q)$ una función doblemente diferenciable en $U\subseteq\mathbb{R}^{2d}$. Luego, para cada $t$ fijo, el mapeo $\varphi_t$ es una transformación simpléctica
  para cualquier condición inicial $y_o$.
\end{theorem}
\begin{proof}
 Para este caso, tenemos $g'(p,q) = \dpart{\varphi_t}{y_o}(y_o)$ y, por lo tanto,
 \begin{align*}
  \frac{d}{dt}\left[ \left(\dpart{\varphi_t}{y_o}\right)^T J \left(\dpart{\varphi_t}{y_o}\right) \right]
  &= \left[\frac{d}{dt}\left(\dpart{\varphi_t}{y_o}\right)^T\right] J \left(\dpart{\varphi_t}{y_o}\right) + \left(\dpart{\varphi_t}{y_o}\right)^T J \left[\frac{d}{dt}\left(\dpart{\varphi_t}{y_o}\right)\right] \\
  &= \left[J^{-1} \nabla^2 H(\varphi_t)\dpart{\varphi_t}{y_o}\right]^T J \left(\dpart{\varphi_t}{y_o}\right) + \left(\dpart{\varphi_t}{y_o}\right)^T J \left[J^{-1} \nabla^2 H(\varphi_t)\dpart{\varphi_t}{y_o}\right] \\
  &= \left(\dpart{\varphi_t}{y_o}\right)\nabla^2 H(\varphi_t)\left(J^{-1} \right)^T J \left(\dpart{\varphi_t}{y_o}\right) + \left(\dpart{\varphi_t}{y_o}\right)^T \nabla^2 H(\varphi_t)\dpart{\varphi_t}{y_o} = 0
 \end{align*}
 donde usamos que $J^{-1} = J^T$ y $J^2 = -\mathbb{I}$.

 Por lo tanto, $\left(\dpart{\varphi_t}{y_o}\right)^T J \left(\dpart{\varphi_t}{y_o}\right)$ es constante $\forall t$ y para $t=0$ tenemos $\varphi_0(y_o) = y_o$ por lo que $\dpart{\varphi_0}{y_o}=\mathbb{I}$
 \[ \left(\dpart{\varphi_t}{y_o}\right)^T J \left(\dpart{\varphi_t}{y_o}\right) = \left(\dpart{\varphi_o}{y_o}\right)^T J \left(\dpart{\varphi_o}{y_o}\right) = \mathbb{I}^T J \mathbb{I} = J\]
\end{proof}

Este teorema nos habla de una conservación del volumen en el espacio de fases.
Si tenemos un conjunto de puntos $U_o\in\mathbf{R}^{2d}$ a $t=0$, la evolución temporal mantendrá el volumen $\Omega(M)$.
En la \textbf{Figura \ref{fig:cons_gatito}} vemos un ejemplo de conservación del volumen para la evolución de un péndulo simple.

\begin{figure}[H]
	\centering	%trim={<left> <lower> <right> <upper>}
	\includegraphics[width=0.75\textwidth]{introduccion/conservacion_volumen_fases.png}
	\caption{Conservación del volumen de fases inicial (representado por el gato) a lo largo de la evolución temporal de un péndulo simple.
	Imagen extraída de \cite[pp. 185]{BOOK:SPR_INT}.}
	\label{fig:cons_gatito}
\end{figure}

\subsubsection{Integradores simplécticos}{\label{sec:int_simpl}}

Dado que la simplecticidad es una propiedad fundamental de todo sistema hamiltoniano, resulta razonable plantear que el integrador que utilicemos para evolucionar temporalmente
el sistema también lo sea.
Decimos que un integrador de un paso es simpléctico si el mapeo asociado $y_1=\Phi_h(y_o)$ es simpléctico.

El primer integrador simpléctico de interés y que usaremos ampliamente es \textit{MidPoint Rule} (MPR), cuyo esquema es
\begin{equation}{\label{eq:MPR}}
 y_{n+1} = y_n + hJ^{-1}\nabla H\left(\frac{y_{n+1}+y_n}{2}\right)
\end{equation}

Veamos que es de orden 2 tomando $y(t_o+h)$ como la solución exacta a \eqref{eq:ec_hamilton} con $y(t_o)=y_o$
\[ y(t_o+h) = y_o + hJ^{-1}\nabla H(y_o) + O(h^2) \]
\begin{align*}
y_1 = y_o + hJ^{-1}\nabla H\left(\frac{y_1+y_o}{2}\right) &=  y_o + hJ^{-1}\nabla H\left(y_o + \frac{h}{2}J^{-1}\nabla H\left(\frac{y_1+y_o}{2}\right)\right) \\
&= y_o + hJ^{-1}\nabla H(y_o) + O(h^2)
\end{align*}
\[ y(t_o+h) - y_1 = O(h^2) \]
por lo que resulta de orden mayor a 1.
Sin embargo, el esquema de MPR es \textit{simétrico} dado que se mantiene al hacer el cambio $y_{n+1}\to y_n$, $y_n\to y_{n+1}$ y $h\to -h$.
Como todo integrador simétrico debe tener orden par, entonces MPR resulta de orden 2 (o superior).

Además, podemos probar que este esquema es simpléctico
\[ \dpart{y_{n+1}}{y_n} = 1 + hJ^{-1}\dpart{}{y_n} \nabla H\left(\frac{y_{n+1}+y_n}{2}\right) = 1 + hJ^{-1} \nabla^2H\left(\frac{y_{n+1}+y_n}{2}\right)\frac{1}{2}\left[ 1+\dpart{y_{n+1}}{y_n} \right]\]
\[ \left[ 1 - \frac{h}{2}J^{-1} \nabla^2H\left(\frac{y_{n+1}+y_n}{2}\right) \right]\dpart{y_{n+1}}{y_n} =  1 + \frac{h}{2}J^{-1} \nabla^2H\left(\frac{y_{n+1}+y_n}{2}\right)\]
donde $1\equiv\mathbb{I}$ la identidad de $\mathbb{R}^{2d}$.
Definiendo $E\equiv\frac{h}{2}J^{-1} \nabla^2H\left(\frac{y_{n+1}+y_n}{2}\right)$ podemos despejar $\dpart{y_{n+1}}{y_n}$
\[ \dpart{y_{n+1}}{y_n} = (1-E)^{-1}(1+E) \]

Buscamos ver que $\left(\dpart{y_{n+1}}{y_n}\right)^TJ\left(\dpart{y_{n+1}}{y_n}\right)= J$, pero antes mostraremos una propiedad útil de $J$ y $E$
\[ E^T J = \frac{h}{2}\left( J^{-1} \nabla^2H \right)^T J = \frac{h}{2} \nabla^2H(J^{-1})^T  J = -\frac{h}{2} \nabla^2H = -JE\]
\[ (1 + E^T)^{-1}J = \sum_{n\geq0} \left(E^T\right)^n J = J\sum_{n\geq0} (-1)^nE^n = J(1 - E)^{-1}  \]

Ahora si podemos probar la simplecticidad de MPR
\begin{align*}
\left(\dpart{y_{n+1}}{y_n}\right)^TJ\left(\dpart{y_{n+1}}{y_n}\right) &= (1+E^T)(1-E^T)^{-1}J(1-E)^{-1}(1+E) \\
&= J(1-E)(1+E)^{-1}(1-E)^{-1}(1+E) = J
\end{align*}


Sin embargo, es inmediato notar que el esquema de MPR no es explícito; no es posible obtener $y_{n+1}$ como una combinación de funciones conocidas aplicadas a $y_n$ para un hamiltoniano $H$ arbitrario.
El integrador puede volverse explícito para hamiltonianos particulares (como el oscilador armónico), pero no lo será en general.

Otro método simpléctico ampliamente utilizado es \textit{Velocity-Verlet} que,a demás de tener el mismo orden resulta explícito para hamiltonianos separables
\[ H(\mathbf{q}_1,..,\mathbf{q}_N,\mathbf{p}_1,..,\mathbf{p}_N) = \sum_i \frac{p_i^2}{2m} + U(\mathbf{q}_1,..,\mathbf{q}_N)\]

El esquema tiene dos versiones, pero la más utilizada es
\begin{align*}
 p_{n+1/2} &= p_n - \frac{h}{2}\dpart{H}{q}(p_{n+1/2}, q_n) \\
 q_{n+1} &= q_n + \frac{h}{2}\left( \dpart{H}{p}(p_{n+1/2}, q_n) + \dpart{H}{p}(p_{n+1/2}, q_{n+1}) \right) \\
 p_{n+1} &= p_{n+1/2} - \frac{h}{2}\dpart{H}{q}(p_{n+1/2}, q_{n+1})
\end{align*}
que para estos hamiltonianos separables resulta explícita
\begin{align*}
 p_{n+1/2} &= p_n - \frac{h}{2}\dpart{U}{q}(q_n) \\
 q_{n+1} &= q_n + \frac{h}{m}p_{n+1/2} \\
 p_{n+1} &= p_{n+1/2} - \frac{h}{2}\dpart{U}{q}(q_{n+1}) = p_n - \frac{h}{2} \left( \dpart{U}{q}(q_n) + \dpart{U}{q}(q_{n+1}) \right)
\end{align*}

Sin embargo, si existe dependencia de momentos en el potencial
\[ H(\mathbf{q}_1,..,\mathbf{q}_N,\mathbf{p}_1,..,\mathbf{p}_N) = \sum_i \frac{p_i^2}{2m} + U(\mathbf{q}_1,..,\mathbf{q}_N;\mathbf{p}_1,..,\mathbf{p}_N)\]
este integrador se vuelve implícito.
Esta es una característica común a todos los integradores simplécticos: solo resultan explícitos para hamiltonianos separables.

Esto resulta muy relevante, dado que implica que el uso de Dinámica Molecular para sistemas interactuantes mediante potencial de Pauli requerirá además la implementación
de algún método para resolver ecuaciones implícitas como \eqref{eq:MPR}.
El costo computacional de estas implementaciones resulta prohibitivo para sistemas de muchas partículas, como mostraremos más adelante.

En la \textbf{Figura \ref{fig:comp_integ_gatito}} podemos ver una comparación de la conservación del volumen de fases para un péndulo simple para los distintos integradores, algunos de los cuales compararemos en la sección \ref{sec:choque1D}.

\begin{figure}[H]
	\centering	%trim={<left> <lower> <right> <upper>}
	\includegraphics[width=0.9\textwidth]{introduccion/Comparacion_integradores.png}
	\caption{Comparación de distintos integradores (simplécticos o no) a la hora de conservar el volumen de fases al integrar un péndulo ideal.
		Estas imágenes fueron tomadas de \cite[pp. 188]{BOOK:SPR_INT}}
	\label{fig:comp_integ_gatito}
\end{figure}

\subsection{Método de Metropolis-Montecarlo}

En principio, conocer la función de partición $Z$ otorga acceso a todas las magnitudes termodinámicas de interés.
Para un sistema de $N$ partículas no interactuantes, el hamiltoniano total del sistema puede escribirse como la suma de hamiltonianos de una partícula
\[H(\mathbf{q}_1,..,\mathbf{q}_N;\mathbf{p}_1,..,\mathbf{p}_N) = H_1(\mathbf{q}_1, \mathbf{p}_1) + ... +H_N(\mathbf{q}_N, \mathbf{p}_N)\]
lo cual permite inmediatamente factorizar la función de partición canónica
\[ Z = \int e^{-\beta H(\mathbf{q}_1,..,\mathbf{q}_N;\mathbf{p}_1,..,\mathbf{p}_N)} d^{3N}qd^{3N}p = \prod_{i=1}^N \int e^{-\beta H_i(\mathbf{q};\mathbf{p})} d^{3}qd^{3}p = \prod_{i=1}^N Z_i \]

En general, habrá $n\ll N$ funciones de partición distintas, donde $n$ es la cantidad de tipos de partículas (equivalentemente, la cantidad de hamiltonianos diferentes) y el computo de $Z$ resulta simple.

Para sistemas interactuantes, este método no es aplicable al no ser el hamiltoniano separable; excepto en casos particulares, el cálculo de $Z$ para sistemas interactuantes resulta analíticamente imposible.
Aunque calcular los pesos $e^{-\beta H(\mathbf{q}_1,..,\mathbf{q}_N;\mathbf{p}_1,..,\mathbf{p}_N)}$ para una dada configuración es posible, el cómputo de $Z$ exige barrer el espacio de fases $6N$-dimensional,
cuyo volumen crece exponencialmente con $N$.

Para temperaturas finitas, sin embargo, gran parte de este volumen generará un aporte ínfimo a la función de partición $Z$, al ser estados cuyo peso relativo es bajo.
Por esto mismo, tendrán un aporte ínfimo en los valores medios de las magnitudes termodinámicas de interés.
Por lo tanto, buscamos un método que nos permita evitar por completo el cómputo de $Z$ y tome en cuenta solo las configuraciones relevantes para el sistema a una dada temperatura $T$.
El método de Metropolis-Montecarlo (también conocido como Metropolis-Hastings\cite{Metropolis1953}) nos permite lograr esto mediante el uso de cadenas de Markov.

\subsubsection{Cadenas de Markov}

Las cadenas de Markov son ampliamente utilizadas para modelar múltiples fenómenos (biológicos, económicos, sociales) por su simplicidad y teoría bien desarrollada\cite[pp. 1]{BOOK:DURRET}.
El objetivo de las cadenas de Markov es introducir la memoria en una sucesión de variables aleatorias $(X_n)_{n\in \mathbb{N}}$ de la forma más simple posible.
Esto es, la evolución $X_n \to X_{n+1}$ no depende de los $n-1$ estados previos ni del ``tiempo'' $n$, sino únicamente del estado actual $X_n$.

Con esto en mente, una sucesión de variables aleatorias $(X_n)_{n\in \mathbb{N}}$ es una cadena de Markov a tiempo discreto si
\[ P(X_{n+1}=j | X_n = i, X_{n-1} = i_{n-1}, ..., X_o = i_0) = P(X_{n+1}=j | X_n = i)  \]
donde $j, i, i_{n-1}, ..., i_0$ son estados posibles de la cadena.

Se define la matriz de transición $p(i,j)$ de la cadena de Markov como la probabilidad de alcanzar el estado $j$ partiendo del estado $i$, aprovechando la independencia del tiempo $n$
\[ p(i,j) = P(X_{n+1} = j | X_n = i) \]

En particular, si nos interesa ver las probabilidades de transición múltiples pasos adelante (a $n$ pasos)
\begin{align*}
 p_n(i,j) = P(X_n = j | X_0 = i) &= \frac{P(X_n = j, X_0 = i)}{P(X_0 = i)} \\
 &= \sum_k \frac{P(X_n = j, X_{n-1}=k, X_0 = i)}{P(X_0 = i)} \\
 &= \sum_k P(X_n = j| X_{n-1}=k, X_0 = i)\frac{P(X_{n-1}=k, X_0 = i)}{P(X_0 = i)} \\
 &= \sum_k P(X_n = j| X_{n-1}=k)P(X_{n-1}=k| X_0 = i) \\
 &= \sum_k p(k,j)p_{n-1}(i,k)
\end{align*}

Identificando la suma anterior como el producto de matrices $(p_{n-1} \cdot p)(i,j)$ y usando que $p_1(i,j)=p(i,j)$, resulta inmediato que $p_n(i,j) = p^n(i,j)$.
En esta propiedad reside la simplicidad de las cadenas de Markov.

En particular, nos interesan las cadenas de Markov irreducibles; aquellas en las que todo estado $i$ puede alcanzar el estado $j$ en una cantidad finita de pasos.
Matemáticamente, decimos que una cadena de Markov es irreducible si
\[ \forall i,j \quad \exists n\in \mathbb{N} \text{ tal que } p^n(i,j) \neq 0 \]

El caso más simple de una cadena no irreducible es una con dos conjuntos de estados $A$ y $B$ desconectados entre si.
\[ p =
\begin{pmatrix}
 p_A & 0\\
 0 & p_B
\end{pmatrix} \]
En estos casos, resultaría mucho más sensato tratarla como dos cadenas de Markov de estados $A$ y $B$ con sus respectivas matrices $p_A$ y $p_B$.

El otro caso de cadena no irreducible es aquella con estados absorbentes\cite[pp. 6]{BOOK:DURRET}; aquellos que no pueden transicionar a otros estados.
Decimos que un estado $i$ es absorbente si $p(i,j)=\delta_{ij} \text{ } \forall j$.
Estas cadenas, en principio, pueden utilizarse para modelar ciertos problemas como la ``Ruina del jugador''\cite[pp. 1]{BOOK:DURRET}. 
Este problema representa a un apostador que jugará hasta quedarse sin dinero o ganar un dinero $N$ total.
Cada vez que apuesta, tiene una probabilidad $p$ de ganar 1 y una probabilidad $1-p$ de perder 1, tal que
\[ p(i,j) = \delta_{j,i+1}p+\delta_{j,i-1}(1-p) \qquad 0<i<N \]
\[ p(i,j) = \delta_{i,j} \qquad i=0,N \]
donde claramente los estados $0$ y $N$ son absorbentes.

En el método de Montecarlo, las cadenas de Markov utilizadas son irreducibles dado que tienen lo que se conoce como una \textit{distribución estacionaria}, que describimos a continuación.

\subsubsection{Distribución estacionaria}

Cuando el estado inicial de una cadena de Markov tiene una distribución $q(i)$, podemos usar el resultado anterior para calcular la distribución $n$ pasos después
\[ P(X_n=j) = \sum_i P(X_n=j,X_0=i) =  \sum_i P(X_n=j|X_0=i)P(X_0=i) = \sum_i p^n(i,j)q(i) = (q\cdot p^n)(j)\]
usando el producto habitual de matrices y tomando el vector fila de probabilidades $q$.
Por lo tanto, la distribución de probabilidad a tiempo $n$ se relaciona con la inicial según $q_n = q_0\cdot p^n$ o, recursivamente, $q_{n+1} = q_n\cdot p$.

Resulta natural entonces preguntarse si existirá alguna forma de mantener la distribución constante $q_{n+1} = q_n$.
Esto es equivalente a pedir que $q_n$ sea un autovector de $p$ con autovalor 1.
Para diferenciar, llamaremos $\pi(x)$ a esta distribución estacionaria que cumple $\pi p = \pi$.

Obtener $\pi$ implica resolver las \textit{ecuaciones de balance}
\[ \sum_k \pi(k) p(k, j) = \pi(j) \]

Sin embargo, muchas veces resulta más natural el uso de las \textit{condiciones de balance detallado}
\[ \pi(k) p(k, j) = \pi(j)p(j, k) \qquad \forall j,k \]
que nos arroja inmediatamente las ecuaciones de balance al sumar sobre $k$.
\[ \sum_k \pi(k) p(k, j) = \sum_k \pi(j)p(j, k) = \pi(j) \sum_k p(j,k) = \pi(j) \]

Claramente, la condición de balance detallado resulta más fuerte que la de balance y no necesariamente puede cumplirse.
Sin embargo, cuando existe, puede ser mucho más sencilla de encontrar.
Esto es particularmente cierto para el caso de matrices de transición $p$ ralas (con gran cantidad de ceros), donde las condiciones de balance detallado se simplifican (como en la ``Ruina del jugador'' explicada previamente).

A continuación, enunciaremos dos teoremas relevantes para el método de Metropolis\cite[pp. 26]{BOOK:DURRET}.
El primero asegura la existencia de una única distribución estacionaria como solución a las ecuaciones de balance (pero no necesariamente balance detallado).

\begin{theorem}
 Sea $p$ una matriz de transición finita e irreducible. Luego, existe solución única a las ecuaciones de balance $\pi p = \pi$ con $\sum_k \pi(k) = 1$ y $\pi(k)\geq 0$ $\forall k$
\end{theorem}

Es importante aclarar que la hipótesis de matriz finita es necesaria para poder asegurar la normalización de $\pi$.
Existe una versión del teorema para dimensión infinita, pero no es relevante para nuestro tratamiento actual, donde nos limitaremos a un conjunto de estados finito.
Volveremos sobre esto más adelante.

Pero la existencia de una distribución estacionaria puede no ser suficiente si somos incapaces de calcularla.
Si la cadena de Markov comienza con una distribución inicial $q(i)$ cualquiera, esperaríamos que eventualmente alcance la distribución estacionaria.
El siguiente teorema es análogo a la Ley de los Grandes Números para el caso de cadenas de Markov y nos asegura lo anterior, entre otras cosas.

\begin{theorem}{\label{teo:markov_muestreo}}
 Sea $p$ una matriz de transición irreducible con distribución estacionaria $\pi$ y $f$ una función sobre los estados tal que $\sum_x f(x)\pi(x) < \infty$, entonces
 \[ \lim_{n\to\infty} \frac{1}{n} \sum_{m=1}^n f(X_m) = \sum_x f(x)\pi(x) \]
\end{theorem}

Para obtener la convergencia, basta tomar la función $f(x) = \delta_{xy}$ para algún $y$, de forma tal que $\sum_{m=1}^n f(X_m)$ sea la cantidad de tiempo que el sistema estuvo en $y$.
Por el teorema anterior, esto converge a $\pi(y)$.
Pero este teorema es más general y nos permite obtener distintos observables $f$ de la distribución, propiedad que nos resultará muy útil cuando apliquemos el algoritmo.


\subsubsection{Algoritmo de Metropolis-Montecarlo}{\label{sec:alg_mm}}

El objetivo de este algoritmo es obtener muestras de una dada distribución $\pi$ construyendo una cadena de Markov que la tenga como distribución estacionaria.
Para esto, fabricaremos una matriz de transición $p$ que cumpla las condiciones de balance detallado con $\pi$.

Comenzamos con una cadena de Markov con matriz $q(x,y)$ representando la transición propuesta.
Sin embargo, aceptaremos la transición con probabilidad\cite{Metropolis1953}
\[r(x,y) = \min \left\{ \frac{\pi(y)q(y,x)}{\pi(x)q(x,y)}, 1\right\}\]
de forma tal que la probabilidad de transición final resulta $p(x,y) = q(x,y)r(x,y)$.

Veamos que esta $p$ cumple las ecuaciones de balance detallado.
Supongamos $\pi(y)q(y,x) \leq \pi(x)q(x,y)$ tal que 
\[
r(x,y) = \pi(y)q(y,x)/\pi(x)q(x,y) \quad \text{y} \quad r(y,x) = 1\]
\[ \pi(x) p(x, y) = \pi(x) q(x,y)\frac{\pi(y)q(y,x)}{\pi(x)q(x,y)} = \pi(y)q(y,x) = \pi(y)q(y,x)r(y,x) = \pi(y)p(y,x) \]
El caso opuesto resulta completamente análogo.

Para generar muestras con distribución $p$, basta evolucionar temporalmente la cadena durante un tiempo suficientemente largo hasta que alcance el equilibrio y
aprovechar el teorema \ref{teo:markov_muestreo} para muestrear los valores medios de las magnitudes termodinámicas de interés.

En nuestro caso, tomamos los estados $x = (\mathbf{q}_1,..,\mathbf{q}_N;\mathbf{p}_1,..,\mathbf{p}_N)$ como distintas configuraciones del espacio de fases.
Queremos como distribución estacionaria la predicha por el ensamble canónico \[\pi(x) = \frac{e^{-\beta E_x}}{Z} = \frac{1}{Z}e^{-\beta H(\mathbf{q}_1,..,\mathbf{q}_N;\mathbf{p}_1,..,\mathbf{p}_N)}\]
Como dijimos, este algoritmo nos ahorra el problema de calcular la función de partición $Z$, dado que las probabilidades de transición $p$ dependen del cociente de los $\pi(x)$.

Sin embargo, el espacio de fases $6N$-dimensional es continuo y, por lo tanto, la cantidad de estados $x$ es infinita.
Esto no es particularmente problemático dado que toda simulación computacional fuerza el problema a ser finito.
Por lo tanto, el espacio de fases pasa a ser una red finita $6N$-dimensional y podemos usar todo el aparato anterior.

Podemos elegir esta matriz de transición proponiendo pasos en los que seleccionamos una partícula $i$ con probabilidad $1/N$, moviendo sus coordenadas canónicas a otras dentro de un paralelepípedo
centrado en su valor actual
\[
\left\{\begin{matrix}
\mathbf{q}_i \to  \mathbf{q}_i + \Delta_q \mathbf{a} & & a_k,b_k \text{ variables aleatorias independientes y }  \\
\mathbf{p}_i \to  \mathbf{p}_i + \Delta_p \mathbf{b} & &  \text{uniformes en } \{2\frac{j}{M}-1: j\in\mathbb{Z}, 0 \leq j \leq M\}
\end{matrix}\right.\]
donde $M$ es un entero (suele ser el máximo representable por un \texttt{long int} $M = 2^{31}-1\sim 10^9$).

Por lo tanto, si $x$ e $y$ difieren en las coordenadas de más de una partícula tenemos $q(x,y)=0$.
Si difieren unicamente en las coordenadas de la partícula $i$, tenemos
\[q(x,y) = \frac{\Theta(\Delta_q - \Arrowvert \mathbf{q}_i - \mathbf{q}_i' \Arrowvert_\infty)\Theta(\Delta_p - \Arrowvert \mathbf{p}_i - \mathbf{p}_i' \Arrowvert_\infty)}{N(2M+1)^6}\]
por lo que nuestra elección de $q$ es simétrica con $q(x,y) = q(y,x)$.

Aprovechando la simetría de $q$, la probabilidad de aceptación se simplifica
\[r(x,y) = \min\left\{\frac{\pi(y)}{\pi(x)}, 1\right\} = \min\{e^{-\beta(E_y-E_x)}, 1\}\]
De esta forma, si tenemos que $\Delta E = E_y - E_x \leq 0$ aceptamos inmediatamente. Si resulta $\Delta E > 0$, aceptamos con probabilidad $e^{-\beta\Delta E}$.


\subsection{Teorema $\Pi$ y adimensionalización}{\label{sec:intro_pi}}

Es interesante que podamos explotar la independencia de las leyes naturales respecto del sistema de unidades elegido, pues nos dice que toda relación física puede expresarse en forma adimensional. 
Esta propiedad nos permite realizar un \textit{análisis dimensional} para simplificar problemas, proponer leyes de \textit{scaling} e interpretar resultados experimentales.

En esta formulación, usamos que toda ley física puede expresarse en función de sus parámetros $a_1,...,a_n$ como una curva de nivel
\[ f(a_1,...,a_n) = 0 \]

El teorema de Buckingham\cite[pp. 21-22]{BOOK:KUNDU} (también conocido como teorema $\Pi$) dice que las $n$ variables siempre pueden combinarse para formar 
$(n-r)$ variables adimensionales, donde $r$ es la cantidad de dimensiones del problema. 
De esta manera, podemos reescribir la relación anterior como
\begin{equation}{\label{eq:teo_pi}}
\Phi(\Pi_1,..,\Pi_{n-r}) = 0 \qquad \text{ o } \qquad \Pi_1=\phi(\Pi_2,..,\Pi_{n-r})
\end{equation}

Para problemas de mecánica, suele ser $r=3$ en correspondencia con las 3 unidades fundamentales: masa $M$, distancia $L$ y tiempo $T$.
Si agregamos la termodinámica, se suma la temperatura $\theta$, generalmente acompañada de la constante de Boltzmann $k_B$.
Es por esto que es muy habitual trabajar directamente con la magnitud $k_BT$, cuyas unidades son de energía y pueden relacionarse con las de la mecánica $[k_BT] = ML^2/T^2$.

Es por eso que en general, en las simulaciones de mecánica estadística solemos tomar directamente $k_B=1$ (medimos la temperatura en unidades de energía) y tenemos $r=3$.
Esto significa que no necesitamos analizar las $n$ variables del sistema, sino las $n-3$ variables adimensionales, lo cual es muy valioso al permitirnos acotar el espacio de búsqueda.
Esto es especialmente cierto para $n<10$, donde la reducción de la cantidad de variables en 3 resulta apreciable.
Sistemas con una cantidad $n$ de parámetros muy grande no permiten un muestreo completo de todas las variables, sean $n$ o $n-r$.

Un claro ejemplo de la aplicación de este teorema en mecánica estadística es el caso de un gas de partículas interactuando vía el potencial de Lennard-Jones (LJ)
\[ V_{LJ}(r) = \varepsilon\left( \left( \frac{\sigma}{r} \right)^{12} - \left( \frac{\sigma}{r} \right)^6 \right) \]
cuyo hamiltoniano dependerá de la masa $m$ además de los 2 parámetros de Lennard-Jones $\varepsilon$ y $\sigma$.
Además, tenemos las 2 variables termodinámicas a elección, como por ejemplo la densidad $\rho$ y la temperatura $T$.
Por lo tanto, tenemos 5 parámetros del sistema y 3 unidades, por lo que las variables adimensionales de interés bien pueden ser $T^* = T/\varepsilon$ y $\rho^*=\rho\sigma^3$. 
Estas serán las variables que rigen el comportamiento termodinámico del sistema.
También podemos construir $r^*=r/\sigma$, $p^*=p/\sqrt{m\varepsilon}$ y así con todas las magnitudes de interés.

Esto resulta equivalente a pensar que en lugar de usar metro-kilo-segundo (mks), estamos usando $\sigma - m - \sqrt{\varepsilon/m\sigma^2}$.
En este sistema de unidades, tenemos básicamente $\sigma=1=m=\varepsilon$, lo cual también suele llamarse \textit{unidades reducidas}.
A efectos prácticos, estamos \textit{eliminando} las variables $\sigma,m,\varepsilon$ de nuestro análisis.
Luego, todos los gases de LJ con $\sigma$ y $\varepsilon$ a densidad $\rho$ y temperatura $T$ son equivalentes a un LJ con $\sigma=1=\varepsilon=m$ y
densidad $\rho\sigma^3$ y temperatura $T/\varepsilon$; cambiar los parámetros equivale a reescalar las variables termodinámicas.

Como dijimos esto es una cuestión semántica dado que ambos abordajes son equivalentes y facilitan mucho el análisis de las formas funcionales de las distintas relaciones entre parámetros.
Esto será particularmente útil en las secciones \ref{sec:choque1D} y \ref{sec:pauli_gas}, donde veremos que la cantidad de parámetros involucrados en el análisis es comparable a $r$.


\subsection{Función de distribución radial}{\label{sec:intro_gr}}

En el estudio del potencial de Pauli, resulta interesante analizar si pueden existir fases no gaseosas de un sistema interactuando mediante este potencial.
Sin embargo, las simulaciones vía el algoritmo de Metropolis-Montecarlo tiene una inherente dificultad para muestrear observables asociados a la evolución temporal del sistema.
Esto hace que algunos indicadores como el \textit{coeficiente de Lindemann} resulten difíciles de interpretar a la hora de observar transiciones de fase.
Esto ocurre porque este coeficiente mide las fluctuaciones en la posición de las partículas como una forma de analizar su desplazamiento, lo cual carece de sentido para una simulación de Montecarlo.

Sin embargo, la función de distribución radial $g(r)$ cuantifica la cantidad de pares de partículas a distancia $r$, lo cual es fácilmente obtenible en una simulación de Montecarlo.
Esta función de distribución radial se define según 
\begin{equation}{\label{eq:def_gr}}
 g(r) = \frac{2}{N}\frac{\left< N(r,\Delta r) \right>}{\rho V(r,\Delta r)} \approx \frac{2}{N}\frac{\left< N(r,\Delta r) \right>}{\rho 4\pi  r^2\Delta r}
\end{equation}
donde $N(r,\Delta r)$ es la cantidad de partículas dentro de un cascarón esférico de radio $r$ y grosor $\Delta r$ centrado en otra partícula y $V(r,\Delta r)$ es el volumen de dicho cascarón\cite{BOOK:HAILE}.
La aproximación vale para $\Delta r$ pequeño, lo cual es ciertamente deseable para tener una discretización fina.

Para un sistema gaseoso, esperamos que $g(r)$ tienda a 1 rápidamente, dado que el sistema es caótico y tiende a tener partículas a todas las distancias.
En contraposición, un sistema sólido tendrá un $g(r)$ compuesta por picos en distancias características dependientes de la estructura de la red.
Un sistema líquido tendrá una $g(r)$ intermedia, con ciertas oscilaciones alrededor de 1.
En los 3 casos, debemos tener $g(r)\to 1$ para $r$ suficientemente grande y, para potenciales de núcleo duro, esperamos una $g(r)=0$ para $r\leq r_o$ ($r_o$ es el ``radio'' de cada partícula).
En la \textbf{Figura \ref{fig:gr_fases}} podemos ver estos casos para un gas de Lennard-Jones.

\begin{figure}[H]
	\centering	%trim={<left> <lower> <right> <upper>}
	\includegraphics[width=0.5\textwidth]{introduccion/gr_fases.png}
	\caption{Función de distribución radial para las distintas fases de un gas de Lennard-Jones}
	\label{fig:gr_fases}
\end{figure}

\subsection{Materia Nuclear}{\label{sec:intro_NM}}

La Materia Nuclear (NM) consiste en un sistema termodinámico de nucleones eléctricamente neutros, interactuando unicamente mediante interacción nuclear (sin Coulomb)\cite{Molinelli}.
En general, cuantificamos mediante $\delta=(N_n - N_p)/N$ la asimetría entre la cantidad neutrones $N_n$ y protones $N_p$.
Para $\delta=0$, decimos que el sistema es \textit{simétrico}.
Este parámetro se suma a los habituales $T$ y $\rho$ a la hora de obtener una ecuación de estado $U(T,\rho, \delta)$.

Ciertamente, no debe confundirse NM con los nucleones de los átomos.
La formulación de NM asume sistemas infinitos, extensivos y neutros muy diferentes a los núcleos atómicos.
Esta cuestión dificulta la vinculación directa de los resultados teóricos con los obtenidos experimentalmente.

En el abordaje clásico del problema, se modela la interacción nuclear como un potencial efectivo.
Existen múltiples formulaciones de este potencial, pero aquí nos concentraremos en el conocido como QCNM\cite{Dorso1988} (\textit{Quasi-Classical Nuclear Matter}), cuyo potencial de interacción entre \textit{todos los nucleones} es

\begin{equation}{\label{eq:pot_QCNM}}
 V_N(r) = V_o \left[ \left(\frac{r_1}{r}\right)^{p_1} - \left(\frac{r_2}{r}\right)^{p_2} \right] \left( 1 + e^{\frac{r-d}{a}} \right)^{-1}
\end{equation}

Este potencial fue inspirado por el potencial de Lennard-Jonnes, que resulta atractivo a distancia finita y repulsivo a corta distancia, como se espera de la interacción nuclear.
Sin embargo, para corregir el alcance de este potencial se introdujo un decaimiento exponencial.
La intensidad del potencial se tomó $V_o=25.93$MeV con potencias $p_1=6.2$ y $p_2=3.0$, distancias características $r_1=1.757$fm y $r_2=1.771$fm y modulación exponencial $d=3.35$fm y $a=0.833$fm.
De esta manera, hay un mínimo de potencial en $r\approx 2.138$fm de $-5.722$MeV, como puede apreciarse en la \textbf{Figura \ref{fig:graf_QCNM}}.

\begin{figure}[H]
	\centering
	\includegraphics[width=0.7\columnwidth]{introduccion/QCNM_pot.png}
	\caption{Parametrización QCNM del potencial nuclear. 
	El mínimo de potencial se encuentra a $r_o=2.138$fm con $V_N(r_o)=-5.722$MeV.}
	\label{fig:graf_QCNM}
\end{figure}

El hecho de que este potencial trate de forma idéntica protones y neutrones no es una característica general.
Existen potenciales como el CMD que separan la interacción entre $V_{np}$ y $V_{nn/pp}$ donde la interacción entre nucleones idénticos es puramente repulsiva mientras que la interacción protón-neutrón 
es atractiva a distancia finita y repulsiva a distancias cortas.

En general, los parámetros de todos estos potenciales se ajustan para que la energía y la densidad de saturación coincidan con los valores experimentales $E_{NM}^{sat}=-16$MeV y $\rho_o=0.16$fm$^{-3}$\cite{Dorso1988}.
Estas curvas de energía $E(\rho)$ suelen ser cóncavas, con un único mínimo claro.
Trabajos recientes \cite{Schrader2009,GimenezMolinelli2014, GimenezMolinelli2015} han mostrado que para densidades debajo de la densidad de saturación, la curva pierde su concavidad y el sistema entra en un 
régimen de \textit{pastas nucleares}; configuraciones no homogéneas de las partículas.
En este régimen de pastas, la curva se vuelve cuasi-constante, oscilando alrededor de un valor aproximado $-13$MeV (en $T=0.5$MeV)\cite{Dorso2018}; todas las pastas disponen de la misma energía de ligadura (ver \textbf{Figura \ref{fig:curva_Evsrho_dorso}}).

\begin{figure}[H]
	\centering
	\includegraphics[width=0.6\columnwidth]{introduccion/curva_Evsrho_dorso.png}
	\caption{Energía por nucleón en función de densidad para las simulaciones de materia nuclear de \cite{Dorso2018}.
	En el régimen de pastas para $\rho\leq\rho_o$ la energía se vuelve constante, oscilando alrededor de un valor fijo.}
	\label{fig:curva_Evsrho_dorso}
\end{figure}

Estas estructuras se observan principalmente en simulaciones de \textit{Neutron Star Matter} (NSM), donde se sabe que surgen de la competencia entre interacciones nucleares (como CMD o QCNM) e interacción de Coulomb. 
Para sistemas con interacción nuclear sin Coulomb, sin embargo, se sospecha que pueden deberse a efectos de tamaño finito y/o a las condiciones de contorno periódicas.

En contraposición a NSM, las pastas halladas en NM suelen disponer de una única estructura por celda.
Es por todo esto que habitualmente de las denomina \textit{pseudo-pastas}.
Generalmente, las pastas surgidas de estos potenciales clásicos suelen tener una fase líquida seguida de una fase sólida-cristalina a temperatura baja \cite{Dorso2018,Alcain,Dorso2019}.
 
\section{Choque unidimensional}
\subsection{Formulación del problema}

En esta sección, estudiaremos el problema desde un punto de vista teórico, tomando el potencial de Pauli definido según \eqref{eq:def_int_pauli} y obteniendo sus propiedades analíticas para el caso de 2 partículas moviéndose en una única dimensión.

\subsubsection{Características generales}

En su forma más general, un Hamiltoniano interactuando con un potencial dependiente de momentos $V_P$ tendrá ecuaciones de movimiento

\begin{align*}
\dot{\mathbf{q}}_i &= \dpart{H}{\mathbf{p}_i} = \frac{\mathbf{p}_i}{m} + \dpart{V_P}{\mathbf{p}_i} \equiv \frac{\mathbf{p}_i}{m} - \sum_{j\neq i} \mathbf{G}_{ij} \\
\dot{\mathbf{p}}_i &= -\dpart{H}{\mathbf{q}_i} = -\dpart{V_P}{\mathbf{p}_i} \equiv \sum_{j\neq i} \mathbf{F}_{ij}
\end{align*}

donde definimos las \textit{güerzas} $\mathbf{G}$ análogamente a las fuerzas habituales $\mathbf{F}$.
Para el potencial de Pauli \eqref{eq:def_int_pauli}

\[ V_P(\mathbf{q}_1,\mathbf{q}_2;\mathbf{p}_1,\mathbf{p}_2) = De^{-\frac{1}{2}\left( \frac{|\mathbf{q}_1-\mathbf{q}_2|^2}{q_o^2} +\frac{|\mathbf{p}_1-\mathbf{p}_2|^2}{p_o^2} \right)} \]

las güerzas cumplen $\mathbf{G}_{ij} = -\mathbf{G}_{ji}$ al igual que las fuerzas dado que $V_P$ depende de $\mathbf{p}_i - \mathbf{p}_j$. Estas magnitudes toman la forma

\begin{align*}
\sum_{j\neq i} \mathbf{F}_{ij} &= -D \sum_{i<j} \left(-\frac{1}{2}\right)\dpart{s_{ij}^2}{\mathbf{q}_i}e^{-\frac{1}{2}s_{ij}^2} = D \sum_{i\neq j} \frac{\mathbf{q}_i - \mathbf{q}_j}{q_o^2}e^{-\frac{1}{2}s_{ij}^2} \Longrightarrow \mathbf{F}_{ij} = (\mathbf{q}_i - \mathbf{q}_j)\frac{D}{q_o^2}e^{-\frac{1}{2}s_{ij}^2} \\
\sum_{j\neq i} \mathbf{G}_{ij} &= -D \sum_{i<j} \left(-\frac{1}{2}\right)\dpart{s_{ij}^2}{\mathbf{p}_i}e^{-\frac{1}{2}s_{ij}^2} = D \sum_{i\neq j} \frac{\mathbf{p}_i - \mathbf{p}_j}{p_o^2}e^{-\frac{1}{2}s_{ij}^2} \Longrightarrow \mathbf{G}_{ij} = (\mathbf{p}_i - \mathbf{p}_j)\frac{D}{p_o^2}e^{-\frac{1}{2}s_{ij}^2}
\end{align*}

Según lo esperado, dada la simetría entre $p$ y $q$ en $V_P$, las expresiones resultan completamente análogas frente al intercambio $\mathbf{q}\leftrightarrow\mathbf{p}$.
En nuestro caso particular de choque unidimensional, el Hamiltoniano toma la forma simplificada
\[ H(q_1, p_1; q_2, p_2) = \frac{1}{2m}\left( p_1^2 +p_2^2\right) + De^{-\frac{1}{2}\left(\frac{(p_1-p_2)^2}{p_o^2} + \frac{(q_1 - q_2)^2}{q_o^2}\right)} \]
pero esta expresión puede simplificarse aún más definiendo $x = x_1 - x_2$, $X = x_1 + x_2$ (donde $x$ reemplaza $q$ o $p$ según corresponda), aprovechando que el hamiltoniano no depende de $Q$
\begin{equation}{\label{eq:hamiltoniano_1D}}
H(q, p; P) = \frac{P^2}{4m} +\frac{p^2}{4m} + De^{-\frac{1}{2}\left(\frac{p^2}{p_o^2} + \frac{q^2}{q_o^2}\right)}
\end{equation}

El impulso total $P$ se conserva al ser nula la suma de fuerzas (acción y reacción sigue valiendo) y por lo tanto es una constante determinada por las condiciones iniciales del sistema.
Al ser el hamiltoniano definido a menos de una constante, podemos simplemente descartar el término $P^2/2m$ y continuar nuestro análisis.

Estas nuevas variables resultan más apropiadas, dado que en ellas resultará más clara la existencia (o no) de una región excluida dentro del espacio de fases.
Si el potencial emula bien el Principio de Exclusión de Pauli, deberíamos observar que $q$ y $p$ no pueden anularse simultáneamente.

Es importante aclarar que el cambio de coordenadas $(q_1,q_2;p_1,p_2)\rightarrow (q,Q;p,P)$ así definida no resulta una transformación canónica pues no preserva los corchetes de Poisson originales 
\[ \poisson{q_i}{q_j} = 0 = \poisson{p_i}{p_j} \text{ y } \poisson{q_i}{p_j} = \delta_{ij} \]
como podemos ver a continuación
\begin{align*}
\poisson{q}{Q} &= \poisson{q_1-q_2}{q_1+q_2} =  \poisson{q_1}{q_2} + \poisson{-q_2}{q_1} = 0 \\
\poisson{p}{P} &= \poisson{p_1-p_2}{p_1+p_2} = \poisson{p_1}{p_2} + \poisson{-p_2}{p_1} = 0 \\
\poisson{q}{p} &= \poisson{q_1-q_2}{p_1-p_2} =  \poisson{q_1}{p_1} + \poisson{-q_2}{-p_2} = 2 \\
\poisson{Q}{P} &= \poisson{q_1+q_2}{p_1+p_2} =  \poisson{q_1}{p_1} + \poisson{q_2}{p_2} = 2 \\
\poisson{q}{P} &= \poisson{q_1-q_2}{p_1+p_2} = \poisson{q_1}{p_1} + \poisson{-q_2}{p_2} = 1 - 1 = 0 \\
\poisson{q}{P} &= \poisson{q_1+q_2}{p_1-p_2} = \poisson{q_1}{p_1} + \poisson{q_2}{-p_2} = 1 - 1 = 0
\end{align*}

A pesar de que resulta sencillo corregir esto dividiendo las nuevas variables por $\sqrt{2}$, no nos interesa particularmente que las nuevas variables sigan las ecuaciones de Hamilton, dado que no pretendemos integrarlas. Volveremos sobre esto en la sección \textbf{\ref{sec:teo_fases}}.


\subsubsection{Adimensionalización}{\label{sec:adim_choque1d}}

El Hamiltoniano definido por \eqref{eq:hamiltoniano_1D} dispone de 4 parámetros $m, q_o, p_o, D$ que determinarán su comportamiento cualitativo.
Sin embargo, puede apreciarse que las constantes de $q_o$ y $p_o$ claramente resultan meros factores de escala de la interacción y probablemente no jueguen ningún rol en la forma funcional de los observables del sistema.
Pero no solo $q_o$ y $p_o$ son irrelevantes, dado que usando el teorema $\Pi$ se la sección \ref{sec:intro_pi}, podemos redefinir variables para eliminar un tercer parámetro adicional.
Elegimos la masa $m$ como el tercer parámetro a eliminar y, de esta manera, redefinimos $q$, $p$ y $H$ según
\begin{equation}{\label{eq:unidades_red}}
\tilde{q} = q/q_o \qquad \tilde{p} = p/p_o \qquad \tilde{H} = Hm/p_o^2
\end{equation}

Para diferenciar, usaremos el ``$\sim$'' para las variables reducidas y el ``$*$'' para los parámetros reducidos, de manera que el nuevo Hamiltoniano reducido resulte
\begin{equation}{\label{eq:ham_red_1d}}
\tilde{H}(\tilde{q}, \tilde{p}) = \frac{\tilde{p}^2}{4} + \frac{Dm}{p_o^2}e^{-\frac{1}{2}(\tilde{q}^2+\tilde{p}^2)} \equiv \frac{\tilde{p}^2}{4} + D^* e^{-\frac{1}{2}(\tilde{q}^2+\tilde{p}^2)}
\end{equation}
donde queda en evidencia el rol de $D^* = Dm/p_o^2$ como único parámetro funcional del problema.
En general, cualquier observable reducido $O^*$ del sistema tendrá una dependencia únicamente de $D^*$ ($O^* = O^*(D^*)$).
Es importante aclarar que esta adimensionalización puede aplicarse igualmente a sistemas con más de 2 partículas, razón por la cual utilizaremos este resultado a continuación y a lo largo de casi todo este trabajo.

Para el caso de $\tilde{H}$ vemos que para $D^*\to 0$ recuperamos un Hamiltoniano de partícula libre, cuyo diagrama de fases es bien conocido; rectas horizontales en $(q,p)$.
Por otro lado, para $D^*\to \infty$, el término cinético $p^2/4$ resulta despreciable y tenemos un Hamiltoniano dominado puramente por el potencial de Pauli, cuya simetría es 
esférica en el espacio de fases y su diagrama de fases consiste en círculos centrados en el origen.
En algún punto medio entre estas rectas horizontales de $D^*\to0$ y los círculos de $D^*\to\infty$ estará el diagrama de fases más general de \eqref{eq:ham_red_1d}.

Otro observable de particular relevancia es el \textit{área total excluida} $A$.
Como dijimos, el objetivo del potencial de Pauli es generar una región en el espacio de fases alrededor de cada partícula, impidiendo que estas se superpongan.
En el espacio de fases de $(q,p)$, esta región debería estar centrada en el origen y esperamos una forma similar a una elipse.
Una forma de cuantificar esta región excluida es a través de su área $A$, que vía la adimensionalización podemos escribir como

\begin{equation}{\label{eq:area_red}}
A = q_op_oA^*(D^*) = q_op_oA^*\left( \frac{Dm}{p_o^2} \right)
\end{equation}

Lo primero que podemos observar es la clara asimetría entre $q_o$ y $p_o$ para $A$, introducida por la existencia de un término cinético en \eqref{eq:ham_red_1d}.
Resulta intuitivo pensar que para $D^*\to\infty$, este término cinético debería desaparecer, dando lugar a un Hamiltoniano compuesto únicamente por el potencial de Pauli.
Este Hamiltoniano debe ser invariante ante el cambio $q\leftrightarrow p$ y $q_o\leftrightarrow p_o$, por lo que $A$ debería ser invariante también.
En este caso, esperamos una elipse con ejes proporcionales a $q_o$ y $p_o$ como región prohibida.

Más allá de lo anterior, esperamos que $A$ sea creciente con $D$ dado que este regula la intensidad del potencial.
Para esto, basta pedir que $A^*(D^*)$ sea una función creciente de $D^*$, pero inevitablemente esto le exige ser una función decreciente en $p_o$.
Aún así, resulta esperable que el área excluida resulte creciente tanto en $p_o$ como en $q_o$, dado que regulan su alcance.
Para $q_o$ esto es inmediato pues $A\propto q_o$, pero para $p_o$ necesitamos exigir condiciones adicionales sobre $A^*$

\[ \dpart{A}{p_o} = q_o\left( A^*(Dm/p_o^2) - \frac{2Dm}{p_o^2}A^{*'}(Dm/p_o^2) \right) \geq 0 \Longleftrightarrow A^*(D^*) \geq 2D^* A^{*'}(D^*) \Longleftrightarrow \]
\[ \Longleftrightarrow  \frac{d}{dD^*} \left(\log A^*(D^*) \right) \leq \frac{d}{dD^*} \left(\log \left(\sqrt{D^*}\right) \right)
\Longleftrightarrow \frac{d}{dD^*} \left(\log \left(\frac{A^*(D^*)}{\sqrt{D^*}}\right) \right) \leq 0\]
\[\Longleftrightarrow \frac{d}{dD^*} \left(\frac{A^*(D^*)}{\sqrt{D^*}}\right) \leq 0 \]

Esta condición es de lo más razonable, dado que $\sqrt{D^*}\propto p_o^{-1}$ y, por lo tanto, $A\propto A^*(D^*)/\sqrt{D^*}$.
Por lo tanto, pedir $A$ creciente con $p_o$ implica pedir $A\propto A^*(D^*)/\sqrt{D^*}$ decreciente con $D^*$ (que a su vez es decreciente con $p_o$).
Lo intersante es que pudimos escribir estas dependencias en función del parámetro reducido $D^*$.
En resumen, esperamos que

\begin{equation}{\label{eq:props_area_red}}
\frac{d}{dD^*}A^{*}(D^*)\geq 0 \quad \text{ y } \quad \frac{d}{dD^*} \left(\frac{A^*(D^*)}{\sqrt{D^*}}\right) \leq 0
\end{equation}


\subsubsection{Diagrama de fases}{\label{sec:teo_fases}}

Ahora si, atacaremos el Hamiltoniano reducido de \eqref{eq:ham_red_1d} con el objetivo de obtener el diagrama de fases.
En nuestro caso, teniendo dos solo coordenadas $(\tilde{q},\tilde{p})$, esto consiste en obtener las curvas $\tilde{q}(\tilde{p})$ o $\tilde{p}(\tilde{q})$ a través de las ecuaciones de Hamilton.

Sin embargo, como dijimos, las ecuaciones de Hamilton para $(\tilde{q},\tilde{p})$ no se preservan, pero esto no es un problema.
Independientemente de la transformación, es un hecho que el Hamiltoniano se conserva.
Por lo tanto, podemos simplemente plantear la conservación de $\tilde{H}(\tilde{q},\tilde{p})\equiv E^*$ para obtener las curvas de nivel de $\tilde{H}(\tilde{q},\tilde{p})$.
Claramente, la opción más económica es despejar $\tilde{q}(\tilde{p})$ como
\begin{equation}{\label{eq:qvsp}}
\tilde{q}^2(\tilde{p}) = -\tilde{p}^2 - 2 \log\left( \frac{E^*-\tilde{p}^2/4}{D^*} \right) \equiv -\tilde{p}^2 - 2 \log\left( \frac{p_\infty^2-\tilde{p}^2}{4D^*} \right)
\end{equation}
donde usamos que $E^* = p_\infty^{*2}/4$ para $q\to\infty$ y así reescribir intuitivamente la dependencia.
Justamente, tenemos consistentemente $\tilde{q}^2(\tilde{p}\to\pm p_\infty) = \infty$.
En particular, esto implica que la curva está acotada para $-p_\infty \leq \tilde{p} \leq p_\infty$ mientras que $\tilde{q}$ resulta \textit{a priori} no acotado.

Es importante hacer una pausa aquí para aclarar que estamos interesados justamente en los casos no ligados, razón por la cual exigimos (consistentemente) que $\tilde{q}\to\infty$ para algún $p$. 
Esto viene del hecho de que estamos considerando un choque de dos partículas que vienen del infinito.
Hecha esta salvedad, continuamos.

Como podríamos haber intuido, las curvas de nivel \eqref{eq:qvsp} son invariantes ante reflexiones en $q$ y/o $p$, por lo que nos basta con analizar el caso $\tilde{q}\geq 0$, obteniendo la otra mitad del diagrama de fases por reflexión.

Dado que planteamos este potencial con el objetivo de obtener una región excluida alrededor del $(q,p)=(0,0)$, resulta natural analizar la curva $(\tilde{q},\tilde{p})$ que pasa por el $(0,0)$.
Para esta curva, debe ser $E^* = \tilde{H}(0,0) = D^*$ y
\[ \tilde{q}^2 = -\tilde{p}^2 - 2\log\left( \frac{4D^*-\tilde{p}^2}{4D^*} \right)
= -\tilde{p}^2 - 2\log\left( 1-\frac{\tilde{p}^2}{4D^*} \right) \]

Para que esta curva que pasa por el $(0,0)$ exista, deben existir $(\tilde{q}, \tilde{p})$ arbitrariamente cerca.
Para este $\tilde{p}$, expandimos en serie de Taylor truncada
\[ \tilde{q}^2 = -\tilde{p}^2 - 2\log\left( 1-\frac{\tilde{p}^2}{4D^*} \right) \approx -\tilde{p}^2 + 2\frac{\tilde{p}^2}{4D^*}  = \tilde{p}^2 \left( \frac{1}{2D^*} -1 \right)\]

De aquí vemos que $\tilde{q}^2$ será positivo si y solo si $D^* \leq 1/2$.
En este caso, existe una curva de nivel de $\tilde{H}$ que pasa por el $(0,0)$ y, por lo visto en \eqref{eq:qvsp}, alcanza $(\infty, \pm p_\infty)$.
En caso contrario, la curva de nivel de $\tilde{H}= D^*$ consiste únicamente del punto $(0,0)$ y efectivamente existe una exclusión, una zona inaccesible.

Planteado esto, el próximo paso consiste en analizar la forma de las curvas de nivel $\tilde{q}^2(\tilde{p})$.
Realizamos entonces un estudio de función, comenzando por los extremos de $\tilde{q}^2(\tilde{p})$, aquellos valores de $\tilde{p}$ donde
\[ \dpart{\tilde{q}}{\tilde{p}} = 0 \]

Dado que buscar un máximo de $\tilde{q}$ equivale a buscar un máximo de $\tilde{q}^2$, buscamos los máximos de \eqref{eq:qvsp} por simplicidad.
Derivando, obtenemos

\[\dpart{\tilde{q}^2}{\tilde{p}} = -2\tilde{p} - 2\frac{-2\tilde{p}}{p_\infty^2-\tilde{p}^2}
= \frac{-2\tilde{p}}{p_\infty^2-\tilde{p}^2} \left( p_\infty^2-\tilde{p}^2 - 2 \right) \]

Es inmediato ver que esta derivada diverge en $\tilde{p}^2 = p^2_\infty$ según lo esperado.
En particular, se anula en 

\[ \tilde{p}^2 = p^2_\infty - 2 \quad \text{ y } \quad \tilde{p} = 0 \]

Nuevamente, vemos que estos extremos son simétricos respecto de $\tilde{p}=0$.
Sin embargo, es más interesante notar que si $p_\infty^2 > 2$ (o $E^* > 1/2$) existen 3 extremos.
En caso contrario, solo existe uno.

Para el caso $E^* > 1/2$, dado que $\tilde{q}$ viene desde el infinito en $\pm p_\infty$, es necesario que los extremos $\tilde{p}^2 = p_\infty^2 - 2$ sean mínimos y, por simetría, el extremo $\tilde{p}=0$ debe ser un máximo. Por otro lado, para el caso $E^*\leq 1/2$ solo tenemos un único extremo en $\tilde{p}=0$, que necesariamente debe ser un mínimo.

Esquematicamente, tenemos entonces 2 tipos de curva, aquellas con dos mínimos (para $E^* > 1/2$) y aquellas con un mínimo (para $E^* \leq 1/2$).
Esto, sin embargo, no está completo dado que aún no confirmamos que $\tilde{q}^2$ sea positivo en los mínimos.
Si esto no se cumpliese, los mínimos en cuestión serían espurios, correspondientes a un $\tilde{q}$ imaginario.
Comenzamos con el siempre presente extremo $\tilde{p}=0$
\[ \tilde{q}^2(0) = -2\log \left( \frac{p_\infty^2}{4D^*} \right) = 2\log \left( \frac{D^*}{E^*} \right) \]
donde vemos inmediatamente que $\tilde{q}^2(0)$ es positivo si y solo si $E^*\leq D^*$.
\[ \tilde{q}^2(0) \geq 0 \Longleftrightarrow  E^*\leq D^* \]

Para el caso de los extremos $\tilde{p}^2=p^2_\infty - 2$ tenemos
\[ \tilde{q}^2(p^2_\infty-2) = 2-p_\infty^2 -2\log \left( \frac{2}{4D^*} \right) = 2 - p_\infty^2 + 2\log \left( 2D^* \right)  \]
y, por lo tanto, la condición se reduce a 
\[ \tilde{q}^2(p^2_\infty-2) \geq 0 \Leftrightarrow E^*\leq \frac{1}{2}\left( 1 + \log(2D^*) \right) \]

Tenemos entonces dos cotas para $E^*$ que, llamativamente, coinciden para el caso $D^*=1/2$.
Esto no es casual, dado que analizando ambas funciones vemos que para $D^*\geq1/2$ la segunda cota resulta estrictamente más fuerte que la primera.
Es más, si recordamos que la segunda cota solo tiene sentido para $E^*>1/2$ (pues sino este extremo no existe) reobtenemos
\[ \frac{1}{2} ( 1 + \log(2D^*) ) \geq E^*> \frac{1}{2} \Longleftrightarrow D^* > \frac{1}{2} \]

Por lo tanto, la existencia de ambos mínimos positivos ($E^*>1/2$ y $\tilde{q}^2(\tilde{p}^2=p^2_\infty-2)\geq 0$) se da si y solo si existe un área excluida ($D^*>1/2$).
Dado que son condiciones equivalentes, asumamos $D^*>1/2$ y analicemos las curvas de nivel obtenidas.
En particular, existe un valor de energía crítica para la cual 
\[\tilde{q}^2(p^2_\infty-2; E^*=E^*_c) = 0 \quad \text{ con } \quad E^*_c = (1+\log(2D^*))/2\]

Esto implica $\tilde{q}(p^2_\infty-2)=0$ y, por lo tanto, tenemos una curva que conecta $(\infty, \pm p_\infty)$ con $(0,p^2_\infty-2)$.
Por simetría de reflexión en $q$, tenemos una curva idéntica que conecta $(-\infty, \pm p_\infty)$ con $(0,p^2_\infty-2)$ y, por lo tanto, una única curva que conecta $(-\infty, \pm p_\infty)$ con $(\infty, \pm p_\infty)$.

Esto también es válido $\forall$ $E^*\geq E^*_c$, dado que si $\tilde{q}^2(p_\infty^2-2)\leq 0$ y $\tilde{q}^2(p_\infty^2)\geq 0$, por continuidad debe existir 
$p_c^*$ con $p_\infty^2-2\leq \tilde{p}_c\leq p_\infty^2$  tal que $\tilde{q}^2(\tilde{p}_c) = 0$ y el argumento anterior aplica de forma idéntica.
Físicamente, esto corresponde al caso en que las partículas comienzan acercándose desde el infinito y se atraviesan, alejándose infinitamente de nuevo.

Sin embargo, para $E^*\leq E^*_c$, tenemos $\tilde{q}^2(\tilde{p}) > 0$  $\forall$ $\tilde{p}$ (dado que el mínimo es estrictamente positivo).
Este caso, por el contrario, corresponde a un rebote en el que las partículas se acercan hasta una distancia mínima para luego ser repelidas de nuevo hasta el infinito.
Es interesante notar, sin embargo, que esta distancia mínima se alcanza dos veces, por lo que las partículas se acercan desde el infinito hasta una distancia mínima, se alejan una distancia finita, vuelven a acercarse a la distancia mínima y finalmente se alejan hasta el infinito.

Para el caso $D^*\leq 1/2$, inevitablemente tenemos un solo mínimo en $\tilde{p}=0$ que resulta positivo si y solo si $E^*\leq D^*$.
En el caso particular $D^*=E^*$ tenemos una curva de nivel que viene desde $(\infty,\pm p_\infty)$ y llega hasta $(0,0)$ como habíamos visto previamente.
Por lo tanto, para $E^*\leq D^*$ tenemos curvas cóncavas cuyo mínimo tiende a $0$ a medida que $E^*$ tiende a $D^*$.
Este caso corresponde nuevamente a un rebote.
A diferencia del caso $D^*>1/2$, este caso corresponde a un rebote clásico: las partículas se acercan desde el infinito hasta una distancia mínima y luego se alejan infinitamente.

Para $E^*>D^*$ tenemos que $\tilde{q}^2(0) < 0$ y $\tilde{q}^2(p_\infty^2)\geq 0$.
Análogamente a lo anterior, por continuidad debe existir $0\leq \tilde{p}_c\leq p_\infty^2$ tal que $\tilde{q}^2(\tilde{p}_c) = 0$ y nuevamente tenemos una curva que conecta $(-\infty, \pm p_\infty)$ con $(\infty, \pm p_\infty)$.

En resumen, tenemos 4 casos que dan origen a 3 curvas distintas (ver \textbf{Tabla \ref{tab:casos_curvas}}).
Estas curvas y sus diferencias pueden apreciarse en \textbf{Figura \ref{fig:curvas_teo}}.

\begin{table}[h]
	\centering
	\begin{tabular}{|c|c||c|c|}
		\hline
		\multicolumn{2}{|c||}{$D^*\leq1/2$} & \multicolumn{2}{c|}{$D^*>1/2$} \\ \hline
		    $\qquad E^*\leq D^* \qquad$      &    $\qquad E^*> D^* \qquad$       &    $E^*>\frac{1}{2}(1+\log(2D^*))$        &    $E^*\leq\frac{1}{2}(1+\log(2D^*))$     \\ \hline
		    \textbf{Rebote simple}      &    \multicolumn{2}{c|}{\textbf{Atraviese}}     &     \textbf{Rebote doble}    \\ \hline
	\end{tabular}
	\caption{Tipos de trayectoria posibles en el espacio de fases dependiendo de los parámetros $E^*$ y $D^*$}
	\label{tab:casos_curvas}
\end{table}

\begin{figure}[H]
	\centering					%trim={<left> <lower> <right> <upper>}
	\subfigure{\includegraphics[trim = 7mm 0mm 17mm 10mm, clip,width=0.49\textwidth]{choque1d/curvas_teo_1.png}}
	\subfigure{\includegraphics[trim = 7mm 0mm 17mm 10mm, clip,width=0.49\textwidth]{choque1d/curvas_teo_2.png}}
	\subfigure{\includegraphics[trim = 7mm 0mm 17mm 10mm, clip,width=0.49\textwidth]{choque1d/curvas_teo_3.png}}
	\subfigure{\includegraphics[trim = 7mm 0mm 17mm 10mm, clip,width=0.49\textwidth]{choque1d/curvas_teo_4.png}}
	\caption{Posibles trayectorias en el espacio de fases para los 4 casos posibles de $E^*$ y $D^*$.}
	\label{fig:curvas_teo}
\end{figure}

La existencia de región excluida para $D^*>1/2$ puede apreciarse para las curvas de doble rebote, que parecen estar circunventando el origen.
Como dijimos, a medida que $E^*\to E^*_c = (1+\log(2D^*))/2$ estas curvas acercarán sus mínimos al eje $\tilde{q}=0$ hasta transformarse en curvas de atraviese.
Es justamente para $E^*\to E^*_c $ por debajo donde los mínimos de $q\geq0$ y $q\leq0$ se conectan, encerrando la región excluida.
Por lo tanto, la curva $\tilde{q}^2(\tilde{p}; E^*=E_c^*)$ define el perímetro de esta región y podemos utilizarla para calcular el área total encerrada según
\begin{align*}
A^* = 2\int_{-\sqrt{p_\infty^2-2}}^{\sqrt{p_\infty^2-2}} \tilde{q}(p) dp \Bigg|_{p_\infty^2=4E^*_c}
= 4\int_{0}^{\sqrt{p_\infty^2-2}} \sqrt{-p^2 - 2\log\left( \frac{p_\infty^2-p^2}{2} + \log(2D^*) \right)} dp \Bigg|_{p_\infty^2=4E^*_c}
\end{align*}

Mediante el cambio de variables 
\[x = (p_\infty^2-p^2)/2\]
\[dx = -p dp \Rightarrow dp = -dx/\sqrt{p_\infty^2-2x}\] 
y usando que $p_\infty^2 = 4E^*_c$ y $\log(2D^*) = 2E_c^* - 1$ tenemos finalmente
\begin{equation}{\label{eq:area_int_ex}}
A^* = 4\int_{1}^{2E_c^*} \frac{\sqrt{-4E_c^* + 2x - 2\log(x) + 4E_c^* - 2}}{\sqrt{4E_c^* - 2x}} dx
= 4\int_{1}^{2E_c^*} \frac{\sqrt{x -1 - \log(x)}}{\sqrt{2E_c^* - x}} dx
\end{equation}
donde debemos recordar que $A^*$ es el área reducida, siendo el área total $A = q_op_oA^*$.

\begin{figure}[H]
	\centering		%trim={<left> <lower> <right> <upper>}
	\includegraphics[trim = 10mm 2mm 15mm 12mm, clip, width=0.7\textwidth]{choque1d/AvsD_solo_teo.png}
	\caption{Área de la región excluida en función de $D^*$ obtenida de integrar numéricamente \eqref{eq:area_int_ex}. Para valores altos parece haber una relación lineal entre $A^*$ y $\log D^*$.}
	\label{fig:AvsD_teo}
\end{figure}

Sin embargo, la integral de \eqref{eq:area_int_ex} no puede expresarse como una combinación de funciones conocidas, lo cual nos impide por completo obtener una forma funcional clara.
Aún así, podemos realizar una integración numérica de $A^*$ vía regla de trapecios.
Debemos tener especial cuidado en el extremo superior, donde el integrando diverge.
El resultado de esta integración numérica en un amplio rango de $D^*$ puede apreciarse en la \textbf{Figura \ref{fig:AvsD_teo}}, donde vemos que rápidamente se cumple $A^* \approx \alpha \log(D^*) + \beta$.

Recordando las propiedades \eqref{eq:props_area_red} que esperábamos de $A^*$, es inmediato de la \textbf{Figura \ref{fig:AvsD_teo}} que $A^*$ es creciente en $D^*$.
La segunda propiedad, sin embargo, no resulta tan evidente.
En la \textbf{Figura \ref{fig:AvsD_teo_cota}} podemos apreciar el gráfico de $A^*(D^*)/\sqrt{D^*}$, donde vemos que la función se vuelve decreciente para $D^*\geq D^*_{max}\approx6.082$.
Por lo tanto, tenemos que $A$ es creciente en $p_o$ si y solo si $D^*\geq 6.082$, lo cual nos da otro criterio para elegir parte de estos parámetros. 
\begin{figure}[H]
	\centering
	\includegraphics[trim = 9mm 2mm 15mm 12mm, clip, width=0.7\textwidth]{choque1d/Crecimiento_po.png}
	\caption{Curva de $A^*(D^*)/\sqrt{D^*}$ para confirmar la validez de la cota derecha de \eqref{eq:props_area_red}.
	Podemos ver que la función es decreciente para $D^*\geq D_{max}^* =6.082$, donde $A$ resulta entonces creciente con $p_o$.}
	\label{fig:AvsD_teo_cota}
\end{figure}

Es importante aclarar, de todos modos, que la desigualdad derecha de \eqref{eq:props_area_red}
\[ \frac{d}{dD^*} \left(\frac{A^*(D^*)}{\sqrt{D^*}}\right) \leq 0 \]
era una propiedad \textit{deseable}, pero no realmente necesaria.
Que no se cumpla implica que un aumento en $p_o$ puede generar una reducción en el área de la región excluida, pero esto no resulta particularmente problemático.
La propiedad más relevante del potencial de Pauli es la existencia de un área excluida y la capacidad de poder adaptar $D^*$ para que este área sea la buscada, lo cual ciertamente se cumple.
Si un aumento en $p_o$ genera una reducción del área excluida, es posible aumentar $D$ para corregir esto.

Más adelante retomaremos este análisis del espacio de fases desde un punto de vista computacional, simulando el choque y confirmando nuestros resultados con esta sección.




\subsection{Estudio numérico del espacio de fases}{\label{sec:num_choque_1d}}

Como dijimos, las curvas $q(p)$ del diagrama de fases mantienen el Hamiltoniano $H(q,p)$ constante.
A la hora de simular, un método que conserve la energía del sistema arrojará estas curvas $q(p)$ inmediatamente.
Una simulación de dinámica molecular evolucionando el sistema con las ecuaciones de Hamilton con un integrador simpléctico nos asegura esto, por lo que resulta más natural
que otros métodos donde la energía fluctúa (como Metropolis-Montecarlo).


\subsubsection{Implementación de integradores}{\label{sec:imp_integs}}

Según lo que discutimos en la sección \ref{sec:int_simpl}, dado que el Hamiltoniano \eqref{eq:hamiltoniano_1D} es no separable, no disponemos de integradores simplécticos y explícitos.
Con el objetivo de comparar, decidimos implementar 2 integradores no simplécticos pero explícitos y un integrador simpléctico pero no explícito.

Además, para simplificar la notación definimos \[y = \binom{p}{q} \qquad J = \begin{pmatrix}
0 & \mathbb{I} \\
-\mathbb{I} & 0
\end{pmatrix}
\]

Los integradores implementados son Euler, un Runge-Kutta de orden 2 (RK2) y MidPoint Rule (MPR).
Los esquemas pueden apreciarse junto con su información relevante en la \textbf{Tabla \ref{tab:integradores}}.
La simplecticidad de MPR fue probada en \ref{sec:int_simpl} mientras que la no simplecticidad de Euler y RK2 se encuentra en el \textbf{Apéndice \ref{sec:no_simp}}.

\begin{table}[h]
	\centering
	\begin{tabular}{|c|c|c|c|c|}
		\hline
		\textbf{Integrador} & \textbf{Esquema} & \textbf{Orden} & \textbf{¿Explícito?} & \textbf{¿Simpléctico?} \\ \hline
		Euler & $ y_{n+1} = y_n + hJ^{-1}\nabla H(y_n)$ & $1$ & Si & No \\ \hline
		Runge-Kutta 2 & $y_{n+1} = y_n + hJ^{-1}\nabla H\left(y_n+\frac{h}{2}\nabla H(y_n) \right)$ & $2$ & Si & No \\ \hline
		Midpoint Rule & $y_{n+1} = y_n +  hJ^{-1}\nabla H\left(\frac{y_n+y_{n+1}}{2} \right)$ & $2$ & No & Si \\ \hline
	\end{tabular}
	\caption{Información sobre los integradores utilizados}
	\label{tab:integradores}
\end{table}

Dado que no es explícito, para el integrador MPR implementamos además un método de punto fijo para poder resolver cada paso.
Sumando $y_n$ a cada lado del esquema MPR y dividiendo por 2, podemos redefinir $Z = \frac{y_n+y_{n+1}}{2}$ tal que
\[ Z = y_n + \frac{h}{2}J\nabla H(Z) \equiv F(Z) \]
y resolver esta ecuación de punto fijo con parámetro conocido $y_n$ en \textit{cada iteración} evaluando múltiples veces $F(Z)$.
En principio, puede probarse que $|DF(Z)|<1$ (punto fijo converge) para $h$ suficientemente chico dado que el Hessiano del potencial de Pauli está acotado (es gaussiano).
Sin embargo, la cota para $h$ dependerá de los parámetros $p_o$, $q_o$ y $D$ e incluso podría depender del número de partículas $N$ del sistema.
Por lo tanto, en principio nos conformaremos con saber que tal $h$ existe e intentar encontrarlo según la situación.

En lo que sigue, siempre resolvimos el problema de punto fijo $Z = F(Z)$ utilizando $k=5$ iteraciones, al notar que valores más altos de $k$ no mejoraban apreciablemente la convergencia.
Cabe aclarar, sin embargo, que sistemas con más partículas pueden \textit{a priori} exigir valores mayores de $k$.

\subsubsection{Conservación de la energía}

A modo de comparación, corrimos una simulaciones del choque de 2 partículas en 1D, muestreando el Hamiltoniano (de ahora en más, la energía) a lo largo del recorrido.
Consistentemente con lo que vimos en la sección \ref{sec:teo_fases}, durante el choque de 2 partículas pueden darse 2 situaciones: rebote o atraviese, dependiendo del valor inicial del Hamiltoniano $E$ y el valor de $D$.

Dado que nos interesa trabajar con potenciales de Pauli con región excluida, configuramos el potencial de Pauli con $D = 10000$ y aprovechamos la adimensionalización para tomar $qo = 1 = po = m$ tal que 
$D^*=D>1/2$.
En este caso particular, analizaremos un atraviese.
El sistema comienza con una de las partículas en el origen con impulso nulo (en la posición $(0, 0)$ del espacio de fases) y otra a distancia $6$ e impulso $-4$ (en el $(6, -4)$).

\begin{figure}[H]
	\centering
	\subfigure[Euler]{\includegraphics[trim = 20mm 0mm 15mm 10mm, clip, width=0.325\columnwidth]{choque1d/energia_euler_0,01.png}}
	\subfigure[Runge-Kutta 2]{\includegraphics[trim = 20mm 0mm 15mm 10mm, clip, width=0.325\columnwidth]{choque1d/energia_rk2_0,01.png}}
	\subfigure[Midpoint Rule]{\includegraphics[trim = 20mm 0mm 15mm 10mm, clip, width=0.325\columnwidth]{choque1d/energia_mpr_0,01.png}}
	\caption{Energía durante un choque para un potencial de Pauli con $D = 10000$ y $qo = 1 = po = m$. La trayectoria comenzó en el $(q, p) = (6, -4)$, integrada con paso temporal $h=0.01$}
	\label{fig:energ_choq}
\end{figure}

A modo de ejemplo y para visualizar el problema, en la \textbf{Figura \ref{fig:energ_choq}} puede apreciarse la energía durante un choque para paso temporal $h=0.01$.
Es interesante notar que RK2 tiende a aumentar $E$ mientras que Euler tiende a disminuirla; está clara la mayor efectividad de MPR.

Realizamos estos choques para distintos valores de $h$ (y las mismas condiciones iniciales) y analizamos la fluctuación relativa de energía (siempre positiva para este hamiltoniano).
Los resultados pueden apreciarse en la \textbf{Figura \ref{fig:flucvsh}}, donde se confirma la clara superioridad de MPR a la hora de conservar la energía.
Esto último resulta a priori esperable dado que MPR es un integrador simpléctico; no obstante, al ser resuelta la ecuación con un método de punto fijo, esto pudo haber dejado de ser cierto.
Aún así, vemos un cierto rebote para $h\leq10^{-3}$, mostrando que el error relativo comienza a empeorar para $h$ demasiado pequeño.
Esto es llamativo, pero probablemente se deba a complicaciones a la hora de resolver la ecuación implícita mediante punto fijo; es posible que deba tomarse un número $k$ de iteraciones mayor
para valores tan pequeños de $h$.

\begin{figure}[H]
	\centering
	\includegraphics[trim = 0mm 0mm 15mm 10mm, clip, width=0.6\columnwidth]{choque1d/fluct_vs_h.png}
	\caption{Fluctuaciones de energía en función de paso temporal para distintos integradores.
	El integrador MPR se muestra superior en todo el rango, pero vuelve a crecer para $h\leq10^{-3}$, probablemente debido a errores al resolver la ecuación implícita.}
	\label{fig:flucvsh}
\end{figure}

\subsubsection{Primeros diagramas de fases}

Como dijimos, al ser un sistema unidimensional $(q,p)$, el diagrama de fases queda definido univocamente por $H(q,p)\equiv E = \text{cte}$.
Por lo tanto, la conservación de la energía por parte de los integradores resulta suficiente para obtener estas curvas $q(p)$ y así construir el diagrama.
Con este objetivo, configuramos el sistema inicialmente con una partícula en el $(0,0)$ y la otra en un valor $(q_i, p_i)$.
Luego, evolucionamos el sistema muestreando los valores de $q = q_1 - q_2$, $p = p_1 - p_2$.

\begin{figure}[H]
	\centering	%trim={<left> <lower> <right> <upper>}
	\includegraphics[trim = 11mm 0mm 20mm 5mm, clip, width=0.495\textwidth]{choque1d/fases_Euler.png}
	\includegraphics[trim = 11mm 0mm 20mm 5mm, clip, width=0.495\textwidth]{choque1d/fases_RK2_marcado.png}
	\includegraphics[trim = 11mm 0mm 20mm 5mm, clip, width=0.6\textwidth]{choque1d/fases.png}
	\caption{Diagrama de fases para $D=10000$, $m=1=p_o=q_o$ y $h=10^{-3}$. Además de la zona inaccesible, se evidencian las curvas de rebote y atraviese según lo previsto en \ref{sec:teo_fases}.
		Sin embargo, los métodos de Euler y RK2 no preservan la simetría ante reflexiones en $q$ y $p$.
		Para RK2 están marcadas las distancias en $q$ de la curva de rebote más cercana}
	\label{fig:ej_diag_fases}
\end{figure}

Tomamos entonces un rango de condiciones iniciales $(q_i,p_i)=(5,p)$ con $p\in [-6,0]$ y analizamos las curvas $q(p)$ resultantes, sabiendo que el cambio de $p$ se traduce en un cambio de $E$.
Mantuvimos $q_i=5$ fijo dado que así emulábamos una colisión desde el infinito, dado que podemos considerar que $V_P(5,p)\leq 4\times10^{-6}D \ll D$ es nulo.
Utilizamos los mismos parámetros para el potencial de Pauli ($D=10000$, $m=1=p_o=q_o$) por consistencia.
Hicimos esto una vez para cada integrador para así poder comparar la calidad de los diagramas de fases, siempre con paso $h=10^{-3}$.
En la \textbf{Figura \ref{fig:ej_diag_fases}} podemos ver estas curvas superpuestas en nuestros primeros diagramas de fases.

El caso de Euler es notoriamente malo, donde se aprecian franjas desocupadas que no deberían existir.
Estas franjas indican que el integrador no pudo preservar apropiadamente la energía, dado que el $p^2$ posterior al choque es menor al inicial, lo cual concuerda con lo visto en la \textbf{Figura \ref{fig:energ_choq}}.
El caso de RK2 es menos apreciable, pero ciertamente hay una asimetría entre $q$ y $-q$ dado que los puntos de mayor cercanía de las curvas de rebote no están espejados respecto de $q=0$ como marcamos en la \textbf{Figura \ref{fig:ej_diag_fases}}.
Ciertamente, el integrador MPR genera un diagrama de fases que respeta mejor las simetrías de reflexión en $q$ y $p$, consistente con su capacidad para conservar la energía que vimos en \ref{fig:energ_choq}.

Independientemente de esta simetría, es fundamental notar que el diagrama se compone de dos tipos de curva: rebotes y atravieses.
En particular, los rebotes tienen dos puntos de máxima cercanía, en perfecta concordancia con lo analizado en \ref{sec:teo_fases} dado que tenemos $D^*=10^4>1/2$.
Incluso puede verse que la transición de rebote a atraviese (para MPR) ocurre para $4.625\leq p_i\leq4.75$, consistente con
\[ E_c = \frac{p_i^2}{4} = \frac{1}{2}\left( 1 + \log(2D) \right) \approx 5.45 \Longrightarrow p_i \approx 4.67 \]

Veremos en la sección \ref{sec:area_ex_comp} que no solo se verifica esto sino también el área total excluida, cuyo cómputo tiene sus propias complicaciones.


\subsubsection{Conservación del volumen de fases}

Como discutimos en la sección \ref{sec:trans_simp}, la simplecticidad de las ecuaciones de Hamilton asegura no solo la conservación de la energía, sino la conservación del volumen orientado \eqref{eq:area_orien_simp}.
Esto conlleva, en otras cosas, al Teorema de Liouville y la conservación del volumen de fases.
La evolución de una dada región de condiciones iniciales en el espacio de fases preserva su volumen total.

Por lo tanto, otro criterio para calificar la efectividad de los integradores es la conservación del volumen de fases a lo largo de la evolución.
Para esto, realizamos una simulación tomando un conjunto de valores iniciales $(q_i, p_i)$ y evolucionándolos paralelamente en el tiempo.
Asumimos que cada uno de los puntos era el centro de una elipse rígida de ejes definidos, de forma que la región total asociada al conjunto era la unión de estas elípses.
Así, calculamos para cada tiempo el volumen de esta región total para poder analizar su evolución en el tiempo.

Hicimos esto para curvas de rebote (doble) con 81 condiciones iniciales en un cuadrado de $0.04\times0.04$ alrededor del $(6, -4)$ y los mismos parámetros de antes para Pauli.
Tomamos estas elipses suficientemente grandes como para cubrir completamente este cuadrado inicial, por lo que hubo considerable superposición de volúmenes.
Los resultados pueden apreciarse en la \textbf{Figura \ref{fig:vol_fas}}.

\begin{figure}[h]
	\centering
	\subfigure[h = 0.01]{\includegraphics[trim = 0mm 0mm 15mm 10mm, clip, width=0.32\columnwidth]{choque1d/vol_fas_0,01.png}}
	\subfigure[h = 0.005]{\includegraphics[trim = 0mm 0mm 15mm 10mm, clip, width=0.32\columnwidth]{choque1d/vol_fas_0,005.png}}
	\subfigure[h = 0.001]{\includegraphics[trim = 0mm 0mm 15mm 10mm, clip, width=0.32\columnwidth]{choque1d/vol_fas_0,001.png}}
	\caption{Volumen de un conjunto de puntos a lo largo de una curva de rebote. El pico corresponde al momento de máxima interacción.}
	\label{fig:vol_fas}
\end{figure}

Lo primero que podemos notar es la existencia de un pico, que corresponde los instantes de mayor interacción entre las partículas.
Sin embargo, es importante notar que luego de ello los volúmenes recuperan su valor inicial.
Esto es un buen indicio y ocurre para $h=10^{-3}$ para ambos integradores, pero solo para el MPR en los otros 2 casos.
En un tercer caso con $h=0.1$ (grande) que omitimos, ninguno de los 2 métodos cumplía esto.

Para comprender este pico, nos basta analizar la \textbf{Figura \ref{fig:ej_diag_fases}} y notar que entre rebotes se da la mayor interacción.
En esta región, las curvas de nivel se comprimen, aumentando considerablemente su densidad.
Es por esto que nuestro conjunto de puntos (inicialmente dispuestos en un cuadrado) tiende a alinearse en una figura predominantemente unidimensional, donde resulta claro que el cálculo del área
no es apropiada y genera una sobreestimación de la misma.

Sin embargo, nos resulta suficiente notar que el volumen de fases se conserva luego de cada choque, donde vemos nuevamente que el integrador MPR tiene una mayor robustez,
manteniendo las propiedades de una evolución simpléctica para valores de $h$ mayores comparado a Euler.


\subsection{Área excluida}{\label{sec:area_ex_comp}}

Con el objetivo de obtener $A^*(D^*)$ de la simulación, nuevamente tomaremos $q_o = p_o = m = 1$ de forma que $D^* = D$ y $A=A^*$.
De esta manera, solo es necesario calcular el área de la región excluida.
Para esto, sin embargo, es necesario identificar la curva $C$ de rebote que define esta región.
Para simplificar la notación, definimos la curva $C(q_i,p_i)$ como la curva de energía constante $H(q_i,p_i)$; la curva sobre la que evoluciona el sistema.
De esta manera, la curva $C(q_i,p_i)$ puede pensarse como la que describe el sistema con condiciones iniciales $(q_i,p_i)$.

Con esto en mente, corremos simulaciones que muestreen las curvas $C$ para encontrar el mayor valor de $p_i$ tal que $C(q_i, p_i)$ es un rebote.
Como antes, fijaremos $q_i=5$ pero no discriminaremos entre rebotes simples y dobles, dado que moveremos $D^*$ por encima y por debajo de $1/2$.
Realizamos esta búsqueda en forma binaria, tomando un $p_{max}$ que cumpla que $C(q_i, p_{max})$ sea un atraviese, sabiendo que $C(q_i, 0)$ \textbf{\textit{siempre}} es un rebote.
Obtenemos el $p_{max}$ tomando un valor inicial $p_{max} = -3$ y duplicándolo hasta obtener un atraviese.
Con estos 2 valores de impulso, iteramos la búsqueda binaria hasta que $|p_{max}-p_{min}|\leq tol$.
El valor de $tol$ resulta arbitrario, por lo que tomamos uno dependiente de $D^*$ según $tol \sim \sqrt{D^*}\times 10^{-4}$, pues nos resultó eficiente a lo largo del proceso.
El paso temporal fue $h=10^{-3}$ para $D\leq 1000$, $h=5\times10^{-3}$ para $5000\leq D\leq 50000$ y $h=10^{-4}$ para $D\geq 100000$.

\begin{figure}[H]
	\centering
	\includegraphics[trim = 0mm 0mm 15mm 10mm, clip, width=\columnwidth]{choque1d/AvsD_full_teo.pdf}
	\caption{Área prohibida en función de intensidad de potencial junto con algunos de los espacios de fase asociados.
		Tenemos $A^*=0$ para $D\leq 0.5$ y $A\sim \log{D^*}$ para $D\geq 1$ en perfecto acuerdo con la formulación teórica.}
	\label{fig:AvsD}
\end{figure}

Denominamos $p_b$ al impulso resultado de esta búsqueda, cumpliendo que tenemos un rebote para $p\leq p_b$ y un atraviese para $p>p_b$.
Teniendo la curva $C(q_i, p_b)$, la reescribimos como $q(p)$ e integramos para $p\in[-p_{min},p_{min}]$ donde $q(p_{min})=0$ utilizando una regla de trapecios para obtener el área $A^*/2$ (pues tenemos la mitad izquierda de la región excluida).
Ciertamente, estamos aprovechando las simetrías de reflexión que vimos que MPR poseía.

Los resultados pueden apreciarse en la \textbf{Figura \ref{fig:AvsD}} junto con algunos ejemplos del espacio de fases.
En la misma figura está nuevamente el gráfico del área según el cálculo teórico, que resulta concordante.
La primera observación es que $A=0$ para $D\leq 0.5$, donde las curvas de rebote son cóncavas (rebote simple, $p_{min}=0$) para $q\approx 0$ a diferencia del caso $D\geq 0.5$ donde son convexas
(para rebote doble, como en el ejemplo de la \textbf{Figura \ref{fig:ej_diag_fases}}).
De hecho, puede apreciarse que para $D=0.5$, las curvas son prácticamente verticales, marcando este cambio de convexidad.

\begin{figure}[H]
	\centering
	\includegraphics[trim = 0mm 0mm 15mm 10mm, clip, width=0.75\columnwidth]{choque1d/pinfvsD.png}
	\caption{Impulso crítico de transición (en módulo) rebote-atraviese en función de intensidad de potencial junto con el valor teórico, con el que coincide
	correctamente. La linea punteada marca $D^*=1/2$, donde la forma funcional de $p_b$ cambia.}
	\label{fig:pbvsD}
\end{figure}

Paralelamente, podemos graficar los valores de $p_b$ obtenidos.
En particular, es $E_c = p_b^2/4$, por lo que según vimos en \ref{sec:teo_fases} tiene forma de función partida.
\[p_b(D^*) =\left\{\begin{matrix}
2\sqrt{D^*} & D^*\leq 0,5 \\
\sqrt{2(1+\log(2D^*))} & D^*> 0,5
\end{matrix}\right.\]

La comparación del valor obtenido por la simulación con el teórico se encuentra en la \textbf{Figura \ref{fig:pbvsD}}.
Ambos coinciden dentro del (amplio) rango estudiado, lo cual termina de mostrarnos la efectividad de MPR a la hora de integrar las ecuaciones de movimiento.

\section{Gas de Pauli}
%\input{pauligas.tex}

\section{Nuclear Matter}
%\subsection{Primeras simulaciones}

El objetivo de esta sección es analizar la influencia del potencial de Pauli \eqref{eq:def_int_pauli} en la simulación de NM. 
Dado que trabajaremos nuevamente con $N=1000$ partículas, utilizaremos un método de Metropolis-Montecarlo para realizar la simulación.
Como dijimos, el potencial nuclear de QCNM \eqref{eq:pot_QCNM} no distingue entre protones y neutrones, por lo que a efectos de ese potencial tenemos $N=N_n+N_p$ partículas idénticas.
Con la introducción del potencial de Pauli, separaremos las $N=1000$ partículas según sus componentes de spin e isospin, por lo que $N = N_n^\uparrow+N_n^\downarrow+N_p^\uparrow+N_p^\downarrow$.
En particular, nos interesa un caso simétrico, tomamos $N_n^\uparrow=N_n^\downarrow=N_p^\uparrow=N_p^\downarrow = N/4 = 250\equiv n$.
Dado que la exclusión de Pauli solo puede darse para partículas con la misma componente de spin e isospin, tendremos $N(N-1)/2$ pares de interacción nuclear y $4n(n-1)/2 = N(N/4-1)/2$ 
pares de interacción de Pauli (aproximadamente la mitad). 
Utilizaremos los parámetros de Dorso \eqref{eq:params_dorso} en el potencial de Pauli y una masa $m_n=m_p=938MeV/c^2$.

Las densidades habituales de NM resultan del orden de $0.1$fm$^{-3}$, que para un sistema de $N=1000$ partículas exige cajas de lado $L\sim 20$fm.
Si recordamos que previamente impusimos un $s_{cut}^2=10$ para el potencial se Pauli equivalente a una distancia de $s_{cut}q_o \approx 19$fm, vemos inmediatamente que utilizar un criterio de
mínima imagen resulta inviable al exigir $r_{cut}\leq L/2$; necesitaríamos aumentar la cantidad de partículas a $N\sim8000$ para alcanzar densidades $\rho=0.3$fm$^{-3}$.

Es por esto que decidimos usar las condiciones periódicas del sistema sin el criterio de mínima imagen, permitiendo la interacción simultánea con múltiples imágenes de una misma partícula.
Agregando $l$ copias del sistema en cada una de las direcciones (llamadas \textit{layers}), nos basta que $r_{cut}\leq l.L$ para asegurar que no estamos perdiendo interacciones.
Con el objetivo de tener la menor cantidad de \textit{layers} posibles para minimizar el costo computacional, reducimos la distancia de corte de Pauli a $s_{cut}=6$fm, donde el potencial cae a un $5\%$
del valor máximo $D$.
Dado que el potencial nuclear tiene un 
\[r_{cut}^{(N)} = 6\text{fm}\leq r_{cut}^{(P)} = 14.7\text{fm} \]
 nos basta elegir un $l$ que cumpla $l.L \geq 14.7$fm.
Sin embargo, el valor de $L$ dependerá de $\rho$, por lo que si acotamos $\rho\leq 0.3$fm$^{-3}$ vemos que nos basta tomar $l=1$.
Por lo tanto, una única capa de imágenes bastará para nuestras simulaciones.


\begin{figure}[H]
	\centering
	\subfigure[$\rho=0.005$fm$^{-3}$; ñoquis]{\includegraphics[width=0.3\columnwidth]{NM/pastas_x1/noquis_0,005_x1.png}}
	\hspace{0.03\columnwidth}
	\subfigure[$\rho=0.03$fm$^{-3}$; ñoquis]{\includegraphics[width=0.3\columnwidth]{NM/pastas_x1/noquis_0,03_x1.png}}
	\hspace{0.03\columnwidth}
	\subfigure[$\rho=0.04$fm$^{-3}$; ñoquis-spaghettis]{\includegraphics[width=0.3\columnwidth]{NM/pastas_x1/noquis-fideos_0,04_x1.png}}
	\subfigure[$\rho=0.05$fm$^{-3}$; spaghettis]{\includegraphics[width=0.3\columnwidth]{NM/pastas_x1/fideos_0,05_x1.png}}
	\hspace{0.03\columnwidth}
	\subfigure[$\rho=0.06$fm$^{-3}$; spaghettis-lasagnas]{\includegraphics[width=0.3\columnwidth]{NM/pastas_x1/fideos-lasagna_0,06_x1.png}}
	\hspace{0.03\columnwidth}
	\subfigure[$\rho=0.08$fm$^{-3}$; lasagnas]{\includegraphics[width=0.3\columnwidth]{NM/pastas_x1/lasagna_0,08_x1.png}}
	\subfigure[$\rho=0.125$fm$^{-3}$; túneles]{\includegraphics[width=0.3\columnwidth]{NM/pastas_x1/tuneles_0,125_x1.png}}
	\hspace{0.03\columnwidth}
	\subfigure[$\rho=0.16$fm$^{-3}$; burbujas]{\includegraphics[width=0.3\columnwidth]{NM/pastas_x1/burbuja_0,16_x1.png}}
	\hspace{0.03\columnwidth}
	\subfigure[$\rho=0.175$fm$^{-3}$; gas homogéneo]{\includegraphics[width=0.3\columnwidth]{NM/pastas_x1/gas_0,175_x1.png}}
	\caption{Distintas topologías halladas durante la simulación.
	También se muestra una layer adicional de imágenes para facilitar visualización.}
	\label{fig:pastas_x1}
\end{figure}

Dado que nos interesa el estado fundamental de NM, hicimos simulaciones de Montecarlo para $T=0.5$MeV, comenzando el sistema como una red simple cúbica con impulsos según la distribución de Boltzmann.
Evolucionamos el sistema para múltiples densidades $\rho$ entre $0.005$fm$^{-3}$ y $0.3$fm$^{-3}$ hasta que la energía se estabilizara, para lo cual bastaron $10^5N$ pasos.

Los resultados de estas simulaciones fueron inesperados.
Para empezar, en la \textbf{Figura \ref{fig:pastas_x1}} podemos ver las configuraciones de algunas densidades seleccionadas.
Estas configuraciones se muestran con la primer capa de imágenes para poder apreciar mejor las pastas resultantes.
En los casos de fideos y lasagnas, podemos ver que tenemos más de una estructura por celda, algo no habitual en NM según lo que discutimos en la sección \ref{sec:intro_NM}.
Además, se muestran configuraciones menos definidas en $\rho=0.04$fm$^{-3}$ y $\rho=0.06$fm$^{-3}$, probablemente densidades muy cercanas a la transición entre pastas.
Para $\rho\geq 0.17$fm$^{-3}$ las pastas desaparecen y dan lugar a un gas uniforme. 

Lo más extraño, sin embargo, es la tendencia de los ñoquis a reducir su tamaño a medida que $\rho$ baja. 
Para $\rho=0.03$fm$^{-3}$ el sistema está compuesto de ñoquis de $\sim 100$ partículas, pero al alcanzar $\rho=0.005$fm$^{-3}$ el tamaño descendió a $\sim 10$ partículas.
Esta tendencia decreciente parecería indicar que en el límite de bajas densidades $\rho\to0$ el estado fundamental del sistema resulta el de ñoquis de menos de $10$ partículas, 
lo cual no es consistente con la existencia de núcleos atómicos. 
En particular, implica que la repulsión dada por el potencial de Pauli supera la atracción del potencial nuclear.
Esta ineficiencia energética de acercar nucleones puede darse por 2 razones; bien Pauli tiene una gran intensidad $D$ o un gran alcance $q_o$. 
El primer caso es obvio, pero el segundo implica que si 2 partículas logran estar muy cerca sus impulsos deben diferir considerablemente, lo cual impone un aumento de la energía cinética.

Más allá de la topología, analizamos la energía por partícula del sistema $E$ en función de $\rho$, esperando encontrar un mínimo en $\rho=0.04$fm$^{-3}$ de $\sim -16$MeV.
En la \textbf{Figura \ref{fig:Evsrho_QCNMx1}} podemos ver que esto no es así, dado que hayamos un mínimo de $\sim -47$MeV en $\rho=0.2$fm$^{-3}$.
Esta energía de ligadura resulta el triple de la esperada y desplaza la densidad de saturación un $\sim 25\%$. 
También se encuentran marcados los distintos regímenes de pasta del sistema. 
Además, podemos apreciar la clara concavidad de la curva, aún cuando esperamos que en el régimen de pastas esta tienda a ser constante (ver sección \ref{sec:intro_NM}).

\begin{figure}[H]
	\centering
	\includegraphics[width=0.9\textwidth]{NM/Evsrho_full_x1.png}
	\caption{Energía por nucleón en función de la densidad $\rho$. 
		La curva es aproximadamente cóncava y presenta pastas en las densidades menores a la saturación $\rho_o=0.16$ fm$^{-3}$. 
		En $\rho=0.2$ fm$^{-3}$ se encuentra el mínimo de $-47$MeV, una fuerte sobrestimación. }
	\label{fig:Evsrho_QCNMx1}
\end{figure}

\subsection{Reajuste de parámetros y enfriamiento}

Frente a la sobrestimación de las energías de ligadura de la \textbf{Figura \ref{fig:Evsrho_QCNMx1}}, la respuesta natural fue buscar reducir esta energía.
Para esto, resultó natural reducir la intensidad $V_o$ del potencial nuclear presentado en \ref{sec:intro_NM} a la mitad.
Con el objetivo adicional de estudiar la formación de estas pastas, inicializamos el sistema en $T=5$MeV y lo enfriamos escalonadamente hasta alcanzar los $T=0.5$MeV.
En cada temperatura, muestreamos la energía por partícula para obtener la curva calórica.
En la \textbf{Figura \ref{fig:EvsT_pastas}} podemos apreciar algunas de estas curvas calóricas para densidades específicas; una para cada una de las pastas halladas.
Estas pastas resultaron con una única estructura por celda para spaghettis y lasagnas, más acorde a lo esperado en NM.

En todos los casos, la pasta se originó alrededor de $T=3.5$MeV donde puede apreciarse una región de inestabilidad en los valores de $E$.
Sin embargo, es ciertamente llamativo que no exista salto en la energía entre el estado fundamental de la pasta en $T=0.5$MeV.
Este salto en la energía está asociado a la cristalización de la pasta, lo cual como discutimos en la sección  \ref{sec:gr_pauli_gas}, no resulta sencillo en presencia del potencial de Pauli.

\begin{figure}[H]
	\centering	%trim={<left> <lower> <right> <upper>}
	\includegraphics[trim = 5mm 0mm 10mm 5mm, clip, width=0.48\textwidth]{NM/EvsT_noquis.png}
	\includegraphics[trim = 5mm 0mm 10mm 5mm, clip, width=0.48\textwidth]{NM/EvsT_fideos.png}
	\includegraphics[trim = 5mm 0mm 10mm 5mm, clip, width=0.48\textwidth]{NM/EvsT_lasagna.png}
	\includegraphics[trim = 5mm 0mm 10mm 5mm, clip, width=0.48\textwidth]{NM/EvsT_tuneles.png}
	\caption{Curvas calóricas para las densidades representando un tipo de pasta particular; ñoquis, spaghettis, lasagna y túneles.
	La formación de todas ellas ocurre para $3\text{MeV}\leq T\leq4\text{MeV}$.}
	\label{fig:EvsT_pastas}
\end{figure}

La totalidad de las curvas calóricas pueden apreciarse en la \textbf{Figura \ref{fig:EvsT_QCNMx0,5}}, donde vemos una cierta convergencia para $T\leq 2$MeV.
Para $T=5$MeV, las energías resultan decrecientes en $\rho$ pero a medida que $T$ desciende alcanzan una región de transición entre $T=3$MeV y $T=4$MeV.
Para $T\leq2$MeV, las curvas parecen converger a una misma pendiente, lo cual coincide en la \textbf{Figura \ref{fig:EvsT_pastas}} con la formación de pastas bien definidas. 

\begin{figure}[H]
	\centering
	\includegraphics[width=0.9\textwidth]{NM/EvsT_full.png}
	\caption{Curvas calóricas para las distintas densidades. 
	Aunque inicialmente las energías resultan decrecientes en $\rho$, terminan convergiendo para $T\leq 2$MeV.}
	\label{fig:EvsT_QCNMx0,5}
\end{figure}

Este cambio de comportamiento con la temperatura puede apreciarse en la \textbf{Figura \ref{fig:Evsrho_QCNMx0,5}}, donde vemos la curva $E(\rho)$ para  las distintas temperaturas del sistema.
En gris marcamos la región en la que ocurre la aparición de las pastas para la mayor parte de las densidades; correspondiente a $3$MeV$\leq T\leq4$MeV.
Para los valores de $\rho\sim 0.14$fm$^{-3}$ más cercanos a la desaparición de las pastas, esta franja se corre a temperaturas más bajas. 
Es justamente luego de esta transición en $T\sim 3.5$MeV que las curvas $E(\rho)$ pasan de decrecientes a cuasi-constantes, lo cual es más consistente con la existencia de pastas.

Es también llamativo comparando las \textbf{Figuras \ref{fig:Evsrho_QCNMx1}} y \textbf{\ref{fig:Evsrho_QCNMx0,5}} podemos observar que las pastas correspondientes a los distintos rangos de densidades
no parecen verse muy afectados por la reducción de $V_o$. 
La única excepción a esto es la región de burbuja, que al reducir el $V_o$ aumentó a costa de la región de túneles. 
Además, la densidad de transición de pasta a gas se da ahora para $\rho \approx 0.16$fm$^{-3}$, un valor levemente menor al anterior. 
Ciertamente, la forma cóncava de la \textbf{Figura \ref{fig:Evsrho_QCNMx1}} se pierde por completo, por lo que el sistema no parece tener un mínimo claro.
Este mínimo podría encontrarse analizando $\rho\geq 0.16$fm$^{-3}$, pero nos conformamos con que en $T=0.5$MeV las energías para todas las pastas oscilan los $-11$MeV, consistentemente con lo esperado según la \textbf{Figura \ref{fig:curva_Evsrho_dorso}}.

\begin{figure}[H]
	\centering
	\includegraphics[width=0.9\textwidth]{NM/Evsrho_full.png}
	\caption{Curvas $E$ vs $\rho$ para las distintas temperaturas. 
	La región gris marca el rango de temperaturas en que ocurre la formación de las pastas.
	Es allí también donde se da la transición entre la curva $E$ vs $\rho$ decreciente y la cuasi-constante.
	También se encuentran marcadas con líneas verticales las regiones de densidades correspondientes a cada pasta.}
	\label{fig:Evsrho_QCNMx0,5}
\end{figure}

 
\section{Apéndice}
 \subsection{Distribución de Boltzmann}{\label{ap:boltzmann}}

La distribución de Boltzmann se da para todos los sistemas (interactuantes o no) de partículas distinguibles (o clásicas) cuyos potenciales sean independientes de los momentos.
Estos Hamiltoneanos tienen la forma
\[ H(\mathbf{q}_1,..,\mathbf{q}_N,\mathbf{p}_1,..,\mathbf{p}_N) = \sum_{i=1}^N \frac{p_i^2}{2m_i} + U(\mathbf{q}_1,..,\mathbf{q}_N)\]
donde $p_i = |\mathbf{p}_i|$ es el modulo del momento.

Sabiendo que los ensambles son equivalentes en el límite termodinámico, haremos este desarrollo en ensamble canónico por simplicidad.
En este ensamble, la (densidad de) probabilidad de un dado microestado $x=(\mathbf{q}_1,..,\mathbf{q}_N,\mathbf{p}_1,..,\mathbf{p}_N)$ está definida por

\[ P(x) = \frac{e^{-\beta H(\mathbf{q}_1,..,\mathbf{q}_N,\mathbf{p}_1,..,\mathbf{p}_N)}}{\int e^{-\beta H} d^{3N}qd^{3N}p} \]

Aprovechando la forma del Hamiltoneano, la integral del denominador puede factorizarse

\begin{align*}
\int e^{-\beta H} d^{3N}qd^{3N}p &= \int e^{-\beta \left( \sum_i p_i^2/2m_i + U(\mathbf{q}_1,..,\mathbf{q}_N)\right)} d^{3N}qd^{3N}p \\
&= \left(\prod_{i=1}^N\int e^{-\beta\frac{p^2}{2m_i}} d^{3}p\right) \int e^{-\beta U(\mathbf{q}_1,..,\mathbf{q}_N)} d^{3N}q
\end{align*}

Podemos entonces preguntarnos cual es la probabilidad de que una dada partícula $i$ tenga momento $\mathbf{p}_i$.
Esta probabilidad $f_i(\mathbf{p}_i)$ se obtiene integrando sobre todos los posibles estados de las demás partículas y sobre la posición $\mathbf{q}_i$ de la partícula $i$.
Analogamente a la cuenta anterior, vemos que esto arroja

\[ f_i(\mathbf{p}_i) = e^{-\beta\frac{p_i^2}{2m}}\frac{\left(\prod_{j=1, j\neq i}^N\int e^{-\beta\frac{p^2}{2m_j}} d^{3}p\right) \int e^{-\beta U(\mathbf{q}_1,..,\mathbf{q}_N)} d^{3N}q}
{\left(\prod_{j=1}^N\int e^{-\beta\frac{p^2}{2m_j}} d^{3}p\right) \int e^{-\beta U(\mathbf{q}_1,..,\mathbf{q}_N)} d^{3N}q}
= \frac{e^{-\beta\frac{p_i^2}{2m_i}}}{\int e^{-\beta\frac{p^2}{2m_i}} d^3p} \]

Efectuando la integral del denominador, obtenemos finalmente la probabilidad de un dado impulso $\mathbf{p}_i$.
\begin{equation}{\label{eq:dist_MB_imp}}
f_i(\mathbf{p}_i) = \left(\frac{\beta}{2\pi m_i} \right)^{3/2} e^{-\beta\frac{p_i^2}{2m_i}}
\end{equation}
conocida como distribución de Maxwell-Boltzmann.

Dado que esta probabilidad es isótropa, resultaría natural plantearla en función del módulo del impulso para volverla una distribución de una única variable.
Para esto, debemos integrar en todas las posibles direcciones del vector $\mathbf{p}_i$ además de introducir un jacobiano.
Podemos obtener la nueva distribución aprovechando la normalización y realizando el cambio de variables a esféricas
\[1 = \int f_i(\mathbf{p}_i) d^3p = \int_0^\infty\int_0^{4\pi}\left(\frac{\beta}{2\pi m_i} \right)^{3/2} e^{-\beta\frac{p_i^2}{2m_i}} p^2d\Omega dp
= \int_0^\infty 4\pi\left(\frac{\beta}{2\pi m_i} \right)^{3/2} p_i^2 e^{-\beta\frac{p_i^2}{2m_i}} dp_i\]
por lo que la probabilidad de que la $i$-esima partícula tenga un momento con módulo $p$ es
\[ f_i(p) = 4\pi\left(\frac{\beta}{2\pi m_i} \right)^{3/2} p^2 e^{-\beta\frac{p^2}{2m_i}} \]

Sin embargo, si todas las partículas son diferentes (por ejemplo, tienen masas $m_i$ diferentes) esta distribución tiene poco valor estadístico, pues no nos otorga información
del sistema en su totalidad.
Sin embargo, analizar la probabilidad de que la partícula tenga una dada energía cinética $\varepsilon = p^2/2m_i$ permitirá independizarnos de la masa.
Analogamente a lo anterior, planteamos normalización y utilizamos el cambio de variables $2m_i\varepsilon = p^2$ con $d\varepsilon = pdp/m = \sqrt{2m\varepsilon}dp/m$

\[ 1 = \int_0^\infty f_i(p) dp = \int_0^\infty 4\pi\left(\frac{\beta}{2\pi m_i} \right)^{3/2} e^{-\beta\varepsilon} 2m_i\varepsilon\frac{m_i}{\sqrt{2m_i\varepsilon}} d\varepsilon
= \int_0^\infty 2\pi\left(\frac{\beta}{2\pi m_i} \right)^{3/2} e^{-\beta\varepsilon} (2m_i)^{3/2}\sqrt{\varepsilon} d\varepsilon \]

Por lo tanto, la probabilidad de que la $i$-esima partícula tenga una dada energía cinética $\varepsilon$ es identicamente igual para todas las partículas
\begin{equation}
 f(\varepsilon) = \sqrt{\frac{4\beta^3\varepsilon}{\pi}}e^{-\beta\varepsilon}
\end{equation}
y por lo tanto nos permite obtener una propiedad global del sistema.
Si cada partícula tiene probabilidad $f(\varepsilon)d\varepsilon$ de tener energía cinética $\varepsilon$, entonces la cantidad de partículas con esa energía cinética es $Nf(\varepsilon)d\varepsilon$.
Esta es la distribución de energía cinética del sistema.



\subsection{Cálculo de las funciones de Fermi}

En esta sección, hablaremos del método utilizado para computar las $f_\nu(z)$ con especial interés en el caso $\nu=3/2$ que aparece al buscar $\mu(N,V,T)$
y, en menor medida, el caso $\nu=5/2$ asociado a la presión.

Como dijimos, las funciones de Fermi están definidas según su forma integral, pero pueden expresarse como una serie alternada
\begin{equation}{\label{eq:fermi_func_serie}}
 f_\nu(z) \equiv \frac{1}{\Gamma(\nu)}\int_0^\infty \frac{x^{\nu-1}}{z^{-1}e^x+1} dx = \sum_{l\geq 1} (-1)^{l+1}\frac{z^l}{l^\nu}
\end{equation}
cuya convergencia resulta lenta para $z$ de orden 1 o mayor.
Sin embargo, para toda $f_\nu(z)$ existirá algún $z_\nu < 1$ que nos asegure una convergencia veloz de las sumas parciales
\[ f_\nu(z) \approx \sum_{l=1}^{m} (-1)^{l+1}\frac{z^l}{l^\nu}\]
para $z\leq z_\nu < 1$.
Esto se debe a que podemos acortar burdamente el error por
\[ \left|f_\nu(z) - \sum_{l=1}^{m}(-1)^{l+1}\frac{z^l}{l^\nu}\right| \leq \sum_{l=m+1}^\infty z^l = z^{m+1}\sum_{l=0}^\infty z^l = z^{m+1} \frac{1}{1-z} \]
donde usamos $z\leq z_\nu < 1$ para resolver la suma geométrica.

Es de esta cota de donde obtendremos el $z_\nu$ que nos asegure un error relativo tolerable.
Por ejemplo, para $m=20$ y $\nu = 3/2$, podemos aproximar $f_\nu(z)$ para $z\leq z_\nu = 0.65$ con sumas parciales cuyo error relativo está acotado por
\[\frac{|f_\nu(z) - \sum_{l=1}^{m}(-1)^{l+1}\frac{z^l}{l^\nu}|}{f_\nu(z)} \leq 10^{-3} = 0.1\%\]

Este error es suficientemente bajo como para afirmar que el método es preciso.
Sin embargo, para $z = 0.75$, el error relativo ya escala a un $2\%$ y para $z=0.8$ al $10\%$, por lo que resulta necesario mejorar el método.
Una forma de lograr esto sería aumentando el $m$ de las sumas parciales, pero esto conlleva el problema inherente de que los sumandos son estrictamente crecientes en módulo para $z\geq 1$.
Eventualmente, estos valores pueden ser suficientemente grandes como para incurrir en \texttt{overflows} o cancelaciones catastróficas.

\subsubsection{Algoritmo de Wynn}

Dado que la expresión de las $f_\nu(z)$ como serie de potencias resulta problemática para $z\geq 1$, podemos recurrir a otras formas.
Una de estas es la aproximación de Padé por una función racional (cociente de polinomios)
\[ [L/M]_{f_\nu}(z) \equiv \frac{P_L(z)}{Q_M(z)} \approx f_\nu(z) \]
donde $P_L(z)$ y $Q_M(z)$ son polinomios de grado $L$ y $M$, respectivamente.
Estos polinomios no nos interesan dado que utilizaremos el \textit{algoritmo $\epsilon$ de Wynn} para obtener las aproximaciones diagonales $[p/p]_{f_\nu}(z) = \epsilon(0, 2p)$
donde $\epsilon(n,m)$ está definido recursivamente según
\begin{align*}
 &\epsilon(n, -1) = 0 \\
 &\epsilon(n, 0) = \sum_{l=0}^{n} c_lz^l \\
 &\epsilon(n, m+1) = \epsilon(n+1,m-1) + \frac{1}{\epsilon(n+1,m) - \epsilon(n,m)}
\end{align*}
que, para el caso de las $f_\nu(z)$, son $c_0 = 0$ y $c_l=(-1)^{l+1}/l^\nu$ para $l\geq1$.

Para facilitar la visualización, suelen organizarse los $\epsilon$ en una tabla de doble entrada $n$ y $m$ como se muestra en la \textbf{Figura \ref{fig:wynn_esquema}}.
Allí puede verse claramente que $\epsilon$ deben conocerse para computar $\epsilon(n,m)$.
En particular, en la \textbf{Figura \ref{fig:wynn_ejemplo}} puede verse que (dado que $\epsilon(n,-1)=0$) para computar la aproximación $[p/p]$ basta con computar los $\epsilon(n,m)$ con $0\leq n\leq 2p-m$ y $m\leq 2p$.
Esto corresponde a la mitad superior de la antidiagonal, por lo que la cantidad de cómputos es $(2p)^2/2$.

\begin{figure}[H]
	\centering
	\subfigure[Recurrencia entre elementos]{\includegraphics[width=0.26\columnwidth]{apendice/esquema_wynn.png}{\label{fig:wynn_esquema}}}
	\hspace{0.05\columnwidth}
	\subfigure[Ejemplo del cómputo para el caso $p=1$]{\includegraphics[clip, width=0.68\columnwidth]{apendice/ejemplo_wynn.png}{\label{fig:wynn_ejemplo}}}
	\caption{Arreglo de los coeficientes $\epsilon$ en una tabla para facilitar su cómputo}
	\label{fig:wynn}
\end{figure}

Para una serie de Padé $[p/p]$, el error es $O(z^{2p})$, por lo que tomando $p=10$ podemos obtener un orden similar al de la serie truncada.
Por lo tanto, resultaría razonable utilizar el algoritmo de Wynn para $z\geq z_\nu$, donde (en el caso $\nu=3/2$) empalma con el método de serie truncada con un error
relativo de $\sim10^{-6}$ más que aceptable.

Aunque, en general, las $f_\nu(z)$ no poseen una expresión analítica contra la que podamos comparar la eficacia del algoritmo, el caso particular $\nu=1$ si tiene
expresión analítica: $f_1(z) = \log(z+1)$.
A modo de ejemplo, en la \textbf{Figura \ref{fig:ej_wynn_log}} las aproximaciones a $f_1(z) = \log(z+1)$ mediante el algoritmo de Wynn y sus errores relativos.
Usamos como valor ``exacto'' el arrojado por la función \texttt{log} del paquete \texttt{numpy} de \texttt{Python}.
Puede verse que en el rango $0\leq z\leq 20$, la aproximación con $2p=20$ resulta indistinguible del exacto.

\begin{figure}[H]
	\centering
	\subfigure[Valor absoluto]{\includegraphics[width=0.4\columnwidth]{apendice/ejemplo_wynn_log.png}}
	\hspace{0.05\columnwidth}
	\subfigure[Error relativo]{\includegraphics[width=0.4\columnwidth]{apendice/ejemplo_wynn_log_error.png}}
	\caption{Aproximaciones del método de Wynn a la función $f_1(z) = \log(z+1)$, calculada usando el paquete \texttt{numpy} de \texttt{Python}.
	Podemos ver que rapidamente converge a la función para $p=10$, que resulta indistinguible de la exacta.}
	\label{fig:ej_wynn_log}
\end{figure}

Sin embargo, uno podría preguntarse que nos impide utilizar el algoritmo de Wynn en todo el rango de $z$.
Para $z$ pequeño, dado que las sumas parciales de $\epsilon(n,0)$ convergen rapidamente, la resta que aparece en el demonimador de la definición recursiva de
$\epsilon(n, m+1)$ tiende a ser menor al número de máquina, transformándolo en una división por $0$.

Para $z$ suficientemente grande, las sumas parciales de $\epsilon(n,0)$ resultan dominadas por el término $z^n$, con $0\leq n\leq 2p$ y $p\sim 10$.
Por lo tanto, los $\epsilon(n,0)$ se vuelven muy grandes y problemáticos para trabajar para $z\geq 100$.
Para el caso $\nu=3/2$, observamos las primeras inestabilidades para $z\approx 30$.


\subsubsection{Aproximación de Sommerfeld y forma final}

Para $z$ grande, necesitamos otra forma de computar la $f_\nu(z)$.
Surge la pregunta entonces: ¿existe un $z_m$ suficientemente grande para el cual nos baste conocer $f_\nu(z)$ para $z\leq z_m$?
Esta pregunta surge de recordar el objetivo de partida: encontrar el $z$ que cumple $f_\nu(z)=\frac{N\lambda^3}{gV}\equiv \alpha$.
A priori, $\alpha$ no está acotado superiormente, por lo que necesitamos conocer los $z$ para los cuales $f_\nu(z)\to\infty$.
En particular, estamos interesados en que tanto debemos aumentar $z$ para que $f_\nu(z)$ aumente una dada cantidad.

Para esto, utilizamos el conocido lema de Sommerfeld que nos permite aproximar
\begin{equation}{\label{eq:sommerfeld}}
 \int_0^\infty \frac{\Phi(x)}{e^{x-\eta}+1}dx = \int_0^\eta \Phi(x) dx + \frac{\pi^2}{6}\left( \frac{d\Phi}{dx} \right)\Bigg|_{x=\eta} +
 \frac{7\pi^4}{360}\left( \frac{d^3\Phi}{dx^3} \right)\Bigg|_{x=\eta} + ...
\end{equation}

Para el caso de $f_\nu(z)$, podemos usar este lema con $\Phi(x) = x^{\nu-1}/\Gamma(\nu)$ y $z = e^{\eta}$ (o $\eta = \log z$), obteniendo
\begin{equation}{\label{eq:dirac_sommerfeld}}
f_\nu(z) = \frac{1}{\Gamma(\nu)}\left[ \frac{(\log z)^\nu}{\nu} + \frac{\pi^2}{6}(\nu - 1)(\log z)^{\nu-2} + \frac{7\pi^4}{360}(\nu - 1)(\nu - 2)(\nu - 3)(\log z)^{\nu - 4}
+ O((\log z)^{\nu-6})  \right]
\end{equation}

Esto nos dice que para $z$ suficientemente grande, las $f_\nu(z)$ se comportan como potencias del logaritmo de $z$.
Por lo tanto, el crecimiento de $f_\nu(z)$ es considerablemente lento y obtener valores grandes de $f_\nu(z)$ exige valores muchisimo mayores de $z$.
Por ejemplo, utilizar la aproximación de Sommerfeld para resolver $f_{3/2}(z) = 30$ arroja un $z\approx 10^5$.

Para los valores de $z\geq 20$, utilizamos la aproximación \eqref{eq:dirac_sommerfeld} que empalma en $z=20$ con el algoritmo de Wynn con un error relativo de
$\sim10^{-3}$ aceptable (para $\nu=3/2$).

La ventaja de esto es que para $z\geq 20$ tenemos un método explícito y veloz para evaluar $f_{3/2}(z)$, por lo que podemos utilizar algún método para hallar la
raíz de $g(z) = 0 = f_{3/2}(z) - N\lambda^3/gV$ (en nuestro caso, bisecciones).
Por otro lado, para $0\leq z\leq 20$ construimos una Look-Up Table de $\sim 2500$ puntos, interpolando para obtener los valores de $f_\nu(z)$.
En resumen

\[  \left(\nu = \frac{3}{2}\right) \qquad
 \left\{\begin{matrix}
  0 \leq z \leq 20 & \text{Look-Up Table: } \left\{\begin{matrix}
		    0 \leq z \leq 0.65 & \text{Sumas parciales: } f_{3/2}(z) = \sum_{l=1}^{20} (-1)^{l+1} z^l/l^\nu \\
		    0.65 \leq z \leq 20 & \text{Wynn: } \quad f_{3/2}(z) = \epsilon(0,20)
		    \end{matrix}\right. \\
 z\geq 20 & \text{Sommerfeld: } \qquad \qquad f_{3/2}(z) = \frac{4(\log z)^{3/2}}{3\pi^{1/2}}\left[ 1 + \frac{\pi^2}{8}(\log z)^{-2} - \frac{7\pi^4}{384}(\log z)^{-4}  \right]
 \end{matrix} \right.
\]

Para calcular las $f_{5/2}(z)$ asociadas a la presión, debimos aumentar el $z_\nu$ hasta $0.75$ manteniendo el límite inferior de Sommerfeld.
Esta aproximación de la $f_{5/2}(z)$ tiene una calidad similar a la de $f_{3/2}(z)$, como se ve en la \textbf{Figura \ref{fig:f32_f52}}.

\begin{figure}[H]
	\centering
	\subfigure[Función $f_{3/2}(z)$ utilizada para obtener $\mu(N,V,T)$]{\includegraphics[width=0.4\columnwidth]{apendice/F32.png}}
	\hspace{0.05\columnwidth}
	\subfigure[Función $f_{5/2}(z)$ asociada a la presión del sistema]{\includegraphics[width=0.4\columnwidth]{apendice/F52.png}}
	\caption{Forma final de las $f_\nu(z)$ utilizando la separación en 3 métodos: sumas parciales, Wynn y Sommerfeld. Puede verse que empalman correctamente.}
	\label{fig:f32_f52}
\end{figure}



\subsection{Teorema del virial}{\label{ap:teo_virial}}

El teorema del virial es un resultado general para sistemas con $q$ y $p$ acotados que surge tanto de la mecánica clásica como de la mecánica estadística.
Nos concentraremos en la versión de mecánica estadística deducida para el ensamble canónico y aprovechando la equivalencia de ensambles en el límite termodinámico.
En particular, analizaremos los valores medios de los observables
\begin{align*}
\left< \dpart{H}{q_i}q_i \right> = \frac{1}{Z}\int \dpart{H}{q_i}q_i e^{-\beta H} d^{3N}qd^{3N}p  &= -\frac{1}{Z\beta}\int q_i \dpart{}{q_i}\left(e^{-\beta H}\right) d^{3N}qd^{3N}p \\
&= -\frac{1}{Z\beta}\int \left[ \cancel{\dpart{}{q_i}\left(e^{-\beta H} q_i\right)} - e^{-\beta H} \right] d^{3N}qd^{3N}p \\
&= \frac{1}{Z\beta}\int  e^{-\beta H} d^{3N}qd^{3N}p = k_B T
\end{align*}
donde hemos usado que el sistema se encuentra acotado en $q$ para cancelar la integral con la derivada.
Podemos hacer un procedimiento análogo para $p_i$ y así obtener
\begin{equation}{\label{eq:teo_virial_T}}
  \left< \dpart{H}{q_i}q_i \right> = k_B T = \left< \dpart{H}{p_i}p_i \right>
\end{equation}

Utilizando las ecuaciones de Hamilton, podemos deducir de \eqref{eq:teo_virial_T} la conocida versión de la mecánica clásica
(donde ahora $\left< \bullet \right>$ corresponde a valor medio temporal)
\begin{equation}{\label{eq:teo_virial_mec}}
  \left< \dot{q}_ip_i \right> + \left< \dot{p}_iq_i \right> = 0 = \frac{d}{dt}\left< q_ip_i \right>
\end{equation}

Dado que ya tenemos una forma de obtener $T$, resulta natural considerar una fórmula para obtener $P$.
Para esto utilizaremos \ref{eq:teo_virial_mec} separando $\dot{p}_i = F_i = F_i^{int}+F_i^{ext}$ y sumando sobre todas las coordenadas del sistema tal que
\[ 0 = \left< \sum_{i=1}^{3N} \dot{q}_ip_i + F_i^{int}q_i \right> + \left< \sum_{i=1}^{3N}F_i^{ext}q_i \right>  \]
donde vemos que el segundo término corresponde a una especie de trabajo de las fuerzas externas que equivale a $-3PV$\textbf{CITA HAILE, APENDICE B}.
Por lo tanto, la presión resulta
\begin{equation}{\label{eq:teo_virial_P}}
P = \frac{1}{3V}\left< \sum_{i=1}^{3N} \dot{q}_ip_i + F_i^{int}q_i \right>
\end{equation}

Para el caso de un potencial no dependiente de momentos, el primer término de la sumatoria se reduce a la energía cinética.
Sin embargo, para un sistema con interacción como la de Pauli agregará un término de \textit{g\"uerzas}.


\subsection{Integradores no simplécticos: Euler y Runge-Kutta}{\label{sec:no_simp}}

En este apéndice, demostraremos que los métodos de Euler y un Runge-Kutta de orden 2 explícitos no son simplécticos.
Comenzaremos con el método de Euler de esquema
\[ y_{n+1} = y_n + h J^{-1} \nabla H(y_n) \]
cuya derivada respecto de $y_n$ resulta inmediatamente
\[ \dpart{y_{n+1}}{y_n} = 1 + hJ^{-1}\nabla^2H (y_n)\]
\begin{align*}
 \left( \dpart{y_{n+1}}{y_n} \right)^T J \left( \dpart{y_{n+1}}{y_n} \right) &= \left(1 + h\nabla^2 H(y_n)(J^{-1})^T\right) J \left(1 + hJ^{-1}\nabla^2 H(y_n)\right) \\
 &= \left(1 - h\nabla^2 H(y_n)J^{-1}\right) J \left(1 + hJ^{-1}\nabla^2 H(y_n)\right) \\
 &= J - \cancel{h\nabla^2 H(y_n)} + \cancel{h\nabla^2 H(y_n)} - h^2\nabla^2 H(y_n)J^{-1}\nabla^2 H(y_n) \neq J
\end{align*}

Por otro lado, el método Runge-Kutta de orden 2 con esquema
\[ y_{n+1} = y_n + h J^{-1} \nabla H\left(y_n + \frac{h}{2} J^{-1} \nabla H(y_n) \right) \equiv y_n + h J^{-1} \nabla H\left(y_{n+1/2} \right) \]
tiene una derivada respecto de $y_n$ más complicada
\begin{align*}
\dpart{y_{n+1}}{y_n} &=  1 - h\left[ 1 - \frac{h}{2} \nabla^2 H(y_n)J^{-1} \right]\nabla^2H\left(y_n + \frac{h}{2} J^{-1} \nabla H(y_n)J^{-1} \right) \\
&\equiv 1 - h\left[ 1 - \frac{h}{2} \nabla^2 H(y_n)J^{-1} \right]\nabla^2H\left(y_{n+1/2} \right) 
\end{align*}
\begin{align*}
 &\left( \dpart{y_{n+1}}{y_n} \right)^T J \left( \dpart{y_{n+1}}{y_n} \right) = \left( 1 - h\left[ 1 - \frac{h}{2} \nabla^2 H(y_n)J^{-1} \right]\nabla^2H\left(y_{n+1/2} \right)J^{-1} \right) J \\
 &\qquad \qquad \qquad \left(1 + hJ^{-1}\nabla^2H\left(y_{n+1/2} \right)\left[ 1 + \frac{h}{2} J^{-1} \nabla^2 H(y_n) \right] \right)\\
 &= J + \frac{h^2}{2} \nabla^2 H(y_n)\nabla^2H(y_{n+1/2}) + \frac{h^2}{2} \nabla^2H(y_{n+1/2})\nabla^2 H(y_n) -\\
 &h^2\left[ 1 - \frac{h}{2} \nabla^2 H(y_n)J^{-1} \right]\nabla^2H(y_{n+1/2}) J^{-1}\nabla^2H(y_{n+1/2})\left[ 1 + \frac{h}{2} J^{-1} \nabla^2 H(y_n) \right] \\
 &= J + \frac{h^2}{2} \left[ \nabla^2 H(y_n)\nabla^2H(y_{n+1/2}) +  \nabla^2H(y_{n+1/2})\nabla^2 H(y_n) - 2\nabla^2H\left(y_{n+1/2} \right) J^{-1}\nabla^2H(y_{n+1/2}) \right] \\
 & + \frac{h^3}{2} \left[ \nabla^2H(y_{n+1/2}) J^{-1}\nabla^2H(y_{n+1/2})J^{-1} \nabla^2 H(y_n) - \nabla^2 H(y_n)J^{-1}\nabla^2H(y_{n+1/2}) J^{-1}\nabla^2H(y_{n+1/2})\right] \\
 & + \frac{h^4}{4} \nabla^2 H(y_n)J^{-1}\nabla^2H(y_{n+1/2}) J^{-1}\nabla^2H(y_{n+1/2})J^{-1}\nabla^2 H(y_n)
\end{align*}

Para anular los ordenes cuadrático y cúbico, se requiere basicamente $\nabla^2H(y_n) = \nabla^2H(y_{n+1/2})$, pero el término cuártico en $h$ es no nulo.
Por lo tanto, este método RK2 no es simpléctico.

\subsection{Implementación de interacción de Pauli}

El objetivo fue generar un método para implementar el cálculo de las fuerzas totales en un sistema de $N_{part}$ partículas para un potencial dado. 
Esta función \texttt{forces} debía ser tan general como fuese posible para así evitar problemas a la hora de cambiar el potencial. 
Implementamos estas funciones en \texttt{C} y, posteriormente, comparamos su velocidad de ejecución para distintas optimizaciones del compilador: \texttt{O0}, \texttt{O1}, \texttt{O2}, \texttt{O3} y \texttt{Ofast}. 

La opción \texttt{O0} es la opción por default en la que no hay optimizaciones mientras que \texttt{O1} es la opción con optimizaciones elementales. 
Por otro lado, la optimización \texttt{O2} realiza todas las optimizaciones que no envuelven \textit{space-speed tradeoff} (mayor uso de memoria en pos de aumentar la velocidad). La opción \texttt{O3}, en cambio, 
si realiza este tipo de optimizaciones como el inlineado de funciones y la vectorización de ciclos. 
Finalmente, la opción \texttt{Ofast} hace todo esto junto optimizaciones que no son \textit{standard-compliant} como \texttt{-ffast-math}; librería con funciones matemáticas implementadas para tener una mayor 
velocidad pero no necesariamente arrojar resultados con la exactitud apropiada. 

Teniendo estas funciones implementadas de la mejor manera, podríamos compilarlas en una librería dinámica y llamarlas desde \texttt{Python}.
De esta manera, podiamos aprovechar la velocidad de cómputo de \texttt{C} junto con la versatilidad de una sesión interactiva de \texttt{Python}.

Con esto en mente, planteamos 9 posibles implementaciones de la función \texttt{forces} combinando distintos rasgos. 
En todos los casos (excepto el primero), \texttt{forces} llamaba otras funciones para ejecutar el cálculo de la fuerza (\texttt{pair\_force}) o energía (\texttt{pair\_energ}) de interacción entre 2 partículas. 
Los rasgos que modificamos fueron: 

\begin{quote}
\begin{description}
\item[Precalculo:] El pre-cálculo o no de parámetros relevantes para la interacción tanto en la energía como en la fuerza de 2 partículas

\item[Auxiliares:] Separar el cálculo auxiliar en 2 funciones distintas \texttt{pair\_force} y \texttt{pair\_energ} o tener una única función \texttt{pair\_energ\_force}

\item[Modulo:] Que la función auxiliar que calcula la fuerza de interacción devuelva el vector completo (con sus 3 componentes) o solo el módulo.
\end{description}
\end{quote}

Comparamos todas estas funciones con una función \texttt{forces1} sin funciones auxiliares que, en principio, sería la más veloz. Así, fue posible evaluar que tanta velocidad de ejecución se perdía en pos de 
la encapsulación del potencial. 

\subsubsection{Estabilización del tiempo de ejecución}

Preliminarmente, buscamos el valor de $N_{part}$ a partir del cual el tiempo de ejecución resultaba $\tau\sim N_{part}^2$ dado que la cantidad de interacciones que \texttt{forces} calculaba es 
$N_{part}(N_{part}-1)/2\sim N_{part}^2$, bajo la hipótesis de que el tiempo de cómputo de la interacción entre 2 partículas es independiente de sus posiciones. 
Para asegurar esto último, evitamos la aplicación de un radio de corte $r_c$ para el potencial (en este caso Morse) que lo anule $\forall r\geq r_c$.

Esto lo hicimos calculando el tiempo que tomaba computar $N_{iter}$ veces las fuerzas de un sistema de $N_{part}$ partículas. 
Los resultados se muestran en la \textbf{Figura \ref{fig:TvsNpart}} y permiten extrapolar que para simulaciones cuya duración sea mayor a $1ms$ o $N_{part}\geq200$, el comportamiento es cuadrático en $N_{part}$.

\begin{figure}[h]
	\centering
	\includegraphics[width=0.65\columnwidth]{apendice/implementacion/Estabilizacion_Npart.png}
	\caption{Tiempo de ejecución de $N_{iter}$ \texttt{forces} para $N_{part}$ partículas. Para $N_{part}>200$ el tiempo ya es polinómico (cuadrático) para todo $N_{iter}$. 
	Esto se condice con simulaciones cuya duración supera $1ms$}
	\label{fig:TvsNpart}
\end{figure}

\subsubsection{Comparación: Potencial de Morse}

El primer potencial que implementamos fue el de Morse
\[ V_{M} (r) = D\left[1-e^{-\alpha (r-r_{eq})}\right]^2\]
dado que su cómputo es muy similar al de Pauli, exigiendo el cálculo de una exponencial.

Realizamos 9 implementaciones del mismo, destacando la implementación 1 por ser la más directa y las funciones 5 y 9 por su mayor portabilidad, además de que devolvian el módulo de fuerza.

\begin{quote}
\begin{description}
\item[1] Sin funciones auxiliares.
\item[5] Dos funciones, con parámetros precalculados, devuelve modulo de fuerza.
\item[9] Dos funciones, sin parámetros precalculados, devuelve modulo de fuerza (encapsula totalmente el potencial).
\end{description}
\end{quote}

\begin{multicols}{3}
\begin{lstlisting}
float forces1(...){
	...
	for(i=0;i<N;i++){
		for(j=i+1;j<N;j++){
			...
			calculo 
			de fuerza
			...
			calculo 
			de energia
			...
		}	
	}
}
\end{lstlisting}
\columnbreak
\begin{lstlisting}
float forces5(...){
	...
	for(i=0;i<N;i++){
		for(j=i+1;j<N;j++){
			...
			precalculo de parametros
			...
			pair_force(params,..)
			pair_energ(params,..)
		}	
		...
	}
	...
}

float pair_force(params,..){
...
}

float pair_energ(params,..){
...
}
\end{lstlisting}

\columnbreak
\begin{lstlisting}
float forces9(...){
	...
	for(i=0;i<N;i++){
		for(j=i+1;j<N;j++){
			pair_force(...)
			pair_energ(...)
		}	
		...
	}
	...
}

float pair_force(..){
	calculo de 
	parametros
	...
}

float pair_energ(..){
	calculo de 
	parametros
	...
}

\end{lstlisting}

\end{multicols}


Esta portabilidad permite usar las 2 funciones auxiliares como \textit{test} directamente desde \texttt{Python} y aprovechan que el vector dirección de la fuerza es independiente del potencial. 
Las demás funciones son:

\begin{quote}
\begin{description}
\item[2 y 3] Única función auxiliar \texttt{pair\_energ\_force} con y sin parámetros precalculados.
\item[4] Como \texttt{forces5} pero devolviendo el vector completo
\item[6] Como \texttt{forces9} pero con algunos parámetros precalculados (no todos).
\item[7 y 8] Versiones de \texttt{forces6} y \texttt{forces9} con una única función auxiliar \texttt{pair\_energ\_force}.
\end{description}
\end{quote}

Compilando con las distintas optimizaciones \texttt{O} corrimos las simulaciones para sistemas con $N_{part}=216$. 
Calculamos el tiempo por par de interacción mediante 
\[ \tau_{par}=\frac{2\tau}{N_{part}(N_{part}-1)} =\frac{\tau}{23220} \]
donde $\tau$ era el tiempo total de duración de la simulación, obtenido mediante estadística sobre $1000$ corridas. 
Los resultados pueden apreciarse en la \textbf{Figura \ref{fig:CompTodas}}.

\begin{figure}[h]
	\centering
	\includegraphics[trim = 40mm 20mm 40mm 20mm, clip, width=\columnwidth]{apendice/implementacion/Comp_tiempos_morse_todos.png}
	\caption{Comparación de tiempos para las distintas implementaciones del potencial de morse. 
	La tendencia decreciente de \texttt{O0} a \texttt{Ofast} es clara, excepto excepciones puntuales de \texttt{O2} a \texttt{O3}.}
	\label{fig:CompTodas}
\end{figure}

En general, puede verse una tendencia decreciente a medida que se pasa de \texttt{O0} a \texttt{Ofast}, lo cual resulta esperable. 
Sin embargo, algunas excepciones notables se dan en el paso de \texttt{O2} a \texttt{O3}.

De mayor interés es notar que el compilador \texttt{Ofast} logra que las implementaciones 2, 3, 4 y 5 tengan diferencias menores al $5\%$ entre si y respecto de la implementación 1, 
volviéndolas virtualmente equivalentes.  
Por tanto, estas 4 implementaciones bien pueden ser reemplazadas unicamente por 5, que resulta más portable como dijimos previamente.

Además, viendo que las implementaciones 6 y 7 tienen tiempos comparables a 9, resulta razonable reemplazarlas por 9. 
En el caso de \texttt{forces9} es importante aclarar que, al no tener ningún parámetro precalculado, la encapsulación del potencial es completa. 
Desde el punto de vista de la ingeniería de software \texttt{forces9} sería la óptima. 

Por lo tanto, aislamos las funciones \texttt{forces1}, \texttt{forces5} y \texttt{forces9} para cuantificar la pérdida de velocidad en pos de la portabilidad (encapsulación del potencial) y 
se volvieron a correr las simulaciones, obteniendo los resultados de la \textbf{Figura \ref{fig:CompEsp}}

\begin{figure}[h]
	\centering
	\includegraphics[trim = 10mm 5mm 10mm 5mm, clip, width=0.6\columnwidth]{apendice/implementacion/Comp_tiempos_morse.png}
	\caption{Comparación de las implementaciones de mayor portabilidad (5 y 9) contra la implementación directa 1. 
	Aunque 1 parece equivalente a 5, difiere con 9 en $\sim 40\%$}
	\label{fig:CompEsp}
\end{figure}

Como observamos previamente, la implementación 5 resulta equivalente a la 1 para todos los \texttt{O} (excepto \texttt{O0}). 
Sin embargo, la implementación 9 resulta notoriamente más lenta, con una diferencia relativa respecto a 1 que crece a medida que pasamos de \texttt{O0} a \texttt{Ofast} (donde alcanza un $\sim40\%$). 
Esto último puede deberse a que la separación en 2 funciones obliga a calcular la funcion \texttt{exp} 2  veces, lo cual no ocurre en la implementacion 1 (es directa) ni en la 5 (la exponencial se
pasa como parámetro precalculado). 
Dado que la exponencial no es una función elemental, sus tiempos elevados de calculo pueden causar esta discrepancia.

\subsubsection{Comparación: Potencial de Lennard-Jones}

Con el objetivo de confirmar que la abismal diferencia entre la implementación 9 y la 1 se debe al cálculo de \texttt{exp}, realizamos un análisis similar al anterior para el potencial de Lennard-Jones 
\[ V_{LJ} (r) =  4\varepsilon \left[ \left(\frac{r}{\sigma}\right)^{12} -\left(\frac{r}{\sigma}\right)^{6} \right] \]
que puede calcularse utilizando solamente operaciones algebraicas elementales (multiplicación, suma, resta y división). 
Realizamos las siguientes implementaciones, análogas a 1, 5 y 9 en el caso del potencial de Morse
	
\begin{quote}
\begin{description}
\item[1] Sin funciones auxiliares.
\item[2] Dos funciones, con parámetros precalculados, devuelve modulo de fuerza.
\item[3] Dos funciones, sin parámetros precalculados, devuelve modulo de fuerza (encapsula totalmente el potencial).
\end{description}
\end{quote}

Análogamente, hicimos corridas con cada optimización \texttt{O}, promediando los tiempos con $1000$ corridas. 
Los resultados se encuentran en la \textbf{Figura \ref{fig:CompEsp_LJ}}, donde puede verse el gran salto entre \texttt{O0} y \texttt{O1}, con sus sucesivos saltos de menor orden. 
En particular, entre \texttt{O3} y \texttt{Ofast} no hay ninguna mejora de velocidad. 
Sin embargo, sigue manteniendose la abismal diferencia entre la implementación 9 y la 1 de un $40\%$.

\begin{figure}[h]
	\centering
	\includegraphics[trim = 10mm 5mm 10mm 5mm, clip, width=0.6\columnwidth]{apendice/implementacion/Comp_tiempos_LJ.png}
	\caption{Comparación de las implementaciones de mayor portabilidad contra la implementación directa 1. 
	A diferencia del caso de Morse, la implementación 2 y 3 resultan notoriamente más lentas que la 1.}
	\label{fig:CompEsp_LJ}
\end{figure}

En particular, comparando las \textbf{Figuras \ref{fig:CompEsp}} y \textbf{\ref{fig:CompEsp_LJ}} puede verse que el cálculo de LJ resulta el doble de rápido que el de Morse. 
Esto permitiría pensar que aproximadamente la mitad del tiempo de cómputo se invierte en calcular \texttt{exp}, hecho que oculta la diferencia real entre la implementación 1 y 5 de Morse, 
que ahora pasa a ser $\sim 20\%$. 

\subsubsection{Inlineado de funciones}

Los resultados previos parecen apuntar al hecho de que los tiempos de simulación se ven afectados por el hecho de que la función \texttt{forces} llame funciones auxiliares \texttt{pair\_force} 
y \texttt{pair\_energ}. 
Este tiempo puede reducirse drásticamente utilizando \textit{function inlining}, que en \texttt{C} se reduce a agregar el sufijo \texttt{inline} delante de la declaración de la función. 
Este inlining fuerza al compilador a copiar la función dentro del cuerpo del código en lugar de llamarla durante su ejecución. 

Dado que las funciones están ahora dentro del cuerpo del código, resulta razonable pensar que los optimizaciones \texttt{O} serán capaces de optimizar la función \texttt{forces} en conjunto 
con \texttt{pair\_force} y \texttt{pair\_energ}. 
Por lo tanto, en el caso del potencial de Lennard-Jones sería factible hacer el cálculo de los parámetros precalculables dentro de \texttt{pair\_force} y \texttt{pair\_energ}, con la esperanza 
de que los optimizaciones lo noten y efectivamente los precalculen fuera de ellas.

Hicimos entonces una versión equivalente a \texttt{forces3} de LJ con estos últimos cambios y se la corrió bajo los mismos parámetros de antes, obteniendo los resultados de la 
\textbf{Figura \ref{fig:CompEsp_LJ_inline}}.

\begin{figure}[h]
	\centering
	\includegraphics[trim = 10mm 5mm 10mm 5mm, clip, width=0.6\columnwidth]{apendice/implementacion/Comp_tiempos_LJ_inline.png}
	\caption{Comparación de las implementaciones de Lennard-Jones de mayor portabilidad contra la implementación directa 1 utilizando inlining. 
	Solo la optimización \texttt{Ofast} logra volver las implementaciones 1 y 9 comparables; es el único que realiza los precalculos de variables fuera de \texttt{pair\_force} y \texttt{pair\_energ}}
	\label{fig:CompEsp_LJ_inline}
\end{figure}

Los resultados son muy alentadores, dado que las 3 implementaciones resultan indistinguibles dentro del error bajo la optimización \texttt{Ofast}. 
En particular, puede verse como esta modificación resulta terriblemente perjudicial para los los demás optimizaciones, que claramente no realizan el precálculo. 

Finalmente, realizando un trabajo complemente análogo con el potencial de Morse, obtuvimos los resultados de la \textbf{Figura \ref{fig:CompEsp_morse_inline}}. 
Nuevamente, en el caso de \texttt{Ofast} las 3 implementaciones resultan indistinguibles. 
Sin embargo, puede verse como el cálculo de \texttt{exp} nuevamente enmascara las abismales diferencias entre la implementación 1 y 9 para las demás optimizaciones.

\begin{figure}[h]
	\centering
	\includegraphics[trim = 10mm 5mm 10mm 5mm, clip, width=0.6\columnwidth]{apendice/implementacion/Comp_tiempos_morse_inline.png}
	\caption{Comparación de las implementaciones de Morse de mayor portabilidad contra la implementación directa 1 utilizando inlining. A diferencia del caso de LJ, la diferencia resulta menos notoria, 
	enmascarada por el cálculo de \texttt{exp}}
	\label{fig:CompEsp_morse_inline}
\end{figure}

Con todo este análisis, resulta inmediato elegir la implementación de \texttt{forces} con dos funciones auxiliares \texttt{pair\_force} y \texttt{pair\_energ} inlineadas sin parametros precalculados, 
que devuelvan el modulo de la fuerza. Esta opción ya era la óptima del punto de vista de ingeniería de software y, gracias al previo análisis, resulta equivalente en términos de velocidad.

Con esto tenemos una encapsulación total del potencial, permitiendo la implementación de un \texttt{forces} como marco general que funciona para cualquier par de funciones auxiliares \texttt{pair\_force} y 
\texttt{pair\_energ}. 
Estas funciones auxiliares son las que introducirán efectivamente el potencial dentro del cálculo; para cambiar el potencial basta cambiar estas funciones auxiliares.

\end{document}



\end{document}
