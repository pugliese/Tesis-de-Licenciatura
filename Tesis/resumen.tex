\textbf{\Large Resumen}

\vspace{1cm}

En este trabajo estudiamos los efectos del potencial de Pauli en sistemas de partículas distinguibles.
Este potencial busca emular el Principio de Exclusión de Pauli que impide la ocupación de un estado por más de un único fermion, mediante una repulsión en posición y momento 
que evita la superposición en $\mathbf{q}$ y $\mathbf{p}$ de las partículas.
Comenzamos con un estudio del choque unidimensional de dos partículas, que arroja propiedades muy prometedoras.
La principal es la existencia de un región excluida en el espacio de fases $(\Delta q,\Delta p)$, representando la incapacidad de las partículas de superponerse en $(q,p)$.
Además, obtenemos las condiciones sobre los parámetros del potencial de Pauli para que dicha región exista.
Llegamos a estos resultados mediante el análisis directo del Hamiltoniano y mediante una simulación del choque, integrando las ecuaciones de movimiento.
Dado que el potencial de Pauli es dependiente de momentos, integradores simplécticos como Velocity-Verlet se vuelven implícitos.
Analizaremos uno de estos integradores, mostrando su viabilidad y superioridad respecto a otros integradores explícitos pero no simplécticos.

Luego expandimos el problema a $1000$ partículas y analizaremos la concordancia de este sistema respecto a lo esperado para un gas de fermiones.
Dada la cantidad de cómputo necesaria para integrar simplecticamente, debimos utilizar un método de Metropolis-Montecarlo para estas simulaciones.
Vemos que la distribución de energía cinética difieren de la predicha por Maxwell-Boltzmann (gas ideal clásico) y logran emular las de un gas de fermiones a una dada temperatura $T'$ y 
presión $\rho'$ que difieren de la $T$ y $\rho$ impuestas por la simulación.
Además, mostramos que este sistema a temperatura baja mantiene casi idéntica la región excluida del choque unidimensional y tiene una presión no nula.

Finalmente, agregamos este potencial a sistemas de nucleones interactuando mediante potencial nuclear (modelo QCNM) para estudiar los efectos de la exclusión de Pauli en materia nuclear.
Vemos la aparición de pastas atípicas para la materia nuclear, con múltiples estructuras por celda.
Esto sumado a errores para representar las densidades y energías de saturación nos lleva a modificar los parámetros de la interacción nuclear.
Así, obtenemos resultados en mayor concordancia, tanto en términos energéticos como estructurales de la pasta.