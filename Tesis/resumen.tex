\textbf{\Large Resumen}

\vspace{1cm}

En esta tesis estudiaremos los efectos de introducir el potencial de Pauli en sistemas de partículas distinguibles.
Este potencial tiene como objetivo emular el Principio de Exclusión de Pauli que impide la ocupación de un estado por más de un único fermion, mediante una repulsión en posición y momento 
que evita la superposición en $\mathbf{q}$ y $\mathbf{p}$ de las partículas.
Comenzaremos con un estudio del choque unidimensional de dos partículas, que arroja propiedades muy prometedoras.
La principal es la existencia de un región excluida en el espacio de fases $(\Delta q,\Delta p)$, representando la incapacidad de estas partículas de superponerse simultáneamente en $q$ y $p$.
Además, obtendremos las condiciones sobre los distintos parámetros del potencial de Pauli para que dicha región exista.
Llegaremos a estos resultados mediante el análisis directo del Hamiltoniano y posteriormente mediante una simulación del choque, integrando las ecuaciones de movimiento.
Esta integración numérica se hará mediante un integrador simpléctico pero implícito, mostrando su viabilidad y superioridad respecto a otros integradores explícitos pero no simplécticos.
Luego expandiremos el problema a $1000$ partículas y analizaremos la concordancia de este sistema respecto a lo esperado para un gas de fermiones.
Veremos que las distribuciones de energías cinéticas pueden ajustarse por las correspondientes a un gas de fermiones a distinta temperatura y densidad. 
Esto, sin embargo, será posible para un conjunto de parámetros apropiados para el potencial de Pauli.
Además, mostraremos que este sistema a temperatura baja mantiene casi idéntica la región excluida del choque unidimensional.
Finalmente, agregaremos este potencial a sistemas de materia nuclear para analizar el efecto de la exclusión de Pauli.
Veremos la aparición de pastas nucleares atípicas para la materia nuclear, con múltiples estructuras por celda.
Esto sumado a errores para representar las densidades y energías de saturación nos llevará a modificar los parámetros de la interacción nuclear.
Así, obtendremos resultados en mayor concordancia, tanto en términos energéticos como estructurales de la pasta.