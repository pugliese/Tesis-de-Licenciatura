\documentclass[twoside, 12pt]{article}

\usepackage[sc]{mathpazo} % Use the Palatino font
\usepackage[T1]{fontenc} % Use 8-bit encoding that has 256 glyphs
\linespread{1.5} % Line spacing - Palatino needs more space between lines

%\usepackage[twoside,width=16cm,height=24cm,left=3cm]{geometry}
\usepackage[hmarginratio=1:1,top=20mm,width=20cm,height=23.7cm,columnsep=15pt]{geometry} % Document margins
\usepackage{multicol} % Used for the two-column layout of the document
\usepackage[hang, small,labelfont=bf,up,textfont=it,up]{caption} % Custom captions under/above floats in tables or figures
\usepackage{booktabs} % Horizontal rules in tables
\usepackage{float} % Required for tables and figures in the multi-column environment - they need to be placed in specific locations with the [H] (e.g. \begin{table}[H])
\usepackage{hyperref} % For hyperlinks in the PDF

%----------- Agregados para el caso de ustedes -------------------------------
\usepackage[spanish]{babel}% idioma castellano
\usepackage[utf8]{inputenc}% esto es para poder poner los tildes directamente. Puede que cambie de versión a versión de sistema operativos (más información en http://www.aq.upm.es/Departamentos/Fisica/agmartin/webpublico/latex/FAQ-CervanTeX/FAQ-CervanTeX-6.html )
\usepackage{graphicx} % para insertar figuras
\usepackage{subfigure} % para insertar figuras dentro de figuras
\usepackage{times} % plataforma
\usepackage{amsmath} % --para ecuaciones y algunos símbolos 
\usepackage{wrapfig,lipsum}
\usepackage{listings}
\usepackage{color}

\definecolor{dkgreen}{rgb}{0,0.6,0}
\definecolor{gray}{rgb}{0.5,0.5,0.5}
\definecolor{mauve}{rgb}{0.58,0,0.82}

\lstset{frame=tb,
	language=C++,
	aboveskip=3mm,
	belowskip=3mm,
	showstringspaces=false,
	columns=flexible,
	basicstyle={\small\ttfamily},
	numbers=none,
	numberstyle=\tiny\color{gray},
	keywordstyle=\color{blue},
	commentstyle=\color{dkgreen},
	stringstyle=\color{mauve},
	breaklines=true,
	breakatwhitespace=true,
	tabsize=3
}
% ---------------------- -----------------------------------------------------

\usepackage{lettrine} % The lettrine is the first enlarged letter at the beginning of the text
\usepackage{paralist} % Used for the compactitem environment which makes bullet points with less space between them
\usepackage[T1]{fontenc}					%para poder usar tildes sin problemas

\usepackage{mathrsfs}
% Abreviaturas
%\newcommand\RR{\mathbb{R}}

\graphicspath{{Imagenes/}}

\begin{document}
	
\begin{center}
	{\fontsize{20pt}{10pt}\textbf{Implementación de Morse en pexmd}}
\end{center}

Con el objetivo de probar la implementación del potencial de Morse, corrimos simulaciones para obtener la curva de E(T) y compararla con la obtenida en LAMMPS. Como el método de reescalamiento de velocidades no funcionaba para controlar la temperatura, debimos implementar un termostato de Andersen. 

\section{Morse Interaction Class}

Implementamos el potencial de Morse en \texttt{Python} como una clase heredada de \texttt{ShortRange}

\[ V_{M} (r) = D\left[1-e^{-\alpha (r-r_{eq})}\right]^2\]

Por lo tanto, la clase tiene métodos \texttt{pair\_force}, \texttt{pair\_energ} y \texttt{forces}, utilizando las implementación óptimas \texttt{forces9} en \texttt{C} discutida en \textbf{Compiladores}. En este aspecto, las funciones de \texttt{Python} no son más que wrappers (encapsulamiento) de sus contrapartes en \texttt{C}. Este aspecto es general para las interacciones de tipo \texttt{ShortRange}; lo mismo ocurre en la implementación de Lennard-Jones. 

La clase de \texttt{Python} está definida en \texttt{morse\.py} e importa una librería dinámica \texttt{morse\_pot.so} con las funciones de \texttt{morse\_pot.c}

\section{Prueba del potencial de Morse}

Tomamos parámetros arbitrarios para el potencial de Morse (con partículas de masa $m=1$)
\[ \left\{\begin{matrix}
D = 0.5 & \alpha = 1.0 \\
r_{eq} = 1.0 & r_{cut} = 2.5\\
\end{matrix} \right. \]

Buscamos reproducir la curva $E(T)$ para $1\leq T\leq 10$ a una densidad $\rho = 2$ ($N_{part}=512$, $L=6.3496$).

\subsection{Corrida en LAMMPS}

Dado que en \texttt{pexmd} no teniamos implementadas las condiciones de contorno periódicas, utilizamos inicialmente las \texttt{fixed}. Sin embargo, inmediatamente hubo problemas con partículas escapando de la caja. Esto no ocurría para las condiciones de contorno \texttt{periodic}, así que decidimos utilizarlas. Buscamos entonces que el sistema condense a una única una gota esférica, donde el $r_{cut}$ se encargaría que la gota no interactue con sus imagenes. En este régimen, las condiciones \texttt{fixed} y \texttt{periodic} deberían ser equivalentes. Este régimen se encontró para $\rho=2$ con $N_{part}=512$ y $L=6.3496$.

Usando el termostato de LAMMPS, registramos el par ($E$,$T$) en cada paso. La termalización comenzaba con un aumento hasta $T\sim 15$ donde se estancaba durante un largo tiempo previo a comenzar a bajar. Las pruebas preliminares con $\Delta t= 5\times 10^{-4}$ y $\Delta t= 5\times 10^{-3}$ no lograban alcanzar $T=10$ ni siquiera en $10^5$ pasos.

Por lo tanto, bajamos a $\Delta t = 0.05$ en una corrida de $1.5\times10^{5}$ pasos.
\subsection{Primeras corridas en PEXMD}
\subsection{Termostato de Andersen}
\subsection{Corridas finales en PEXMD}

\section{Conclusiones}

\end{document}